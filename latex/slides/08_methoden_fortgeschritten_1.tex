% The Slide Definitions
\input{../templates/course_definitions}

% Author and Course information
% This Document contains the information about this course.

% Authors of the slides
\author{Marc Satkowski, Sascha Peukert}

% Name of the Course
\institute{C\texttt{\#} Kurs}

% Fancy Logo
\titlegraphic{\hfill\includegraphics[height=1.25cm]{../templates/fsr_logo_cropped}}


% Presentation title
\title{Methoden (fortgeschritten) in C\texttt{\#} - 1}
\date{\today}


\begin{document}

\maketitle

\begin{frame}{Gliederung}
	\setbeamertemplate{section in toc}[sections numbered]
	\tableofcontents
\end{frame}

% ----------------------- Opertatoren Überladung ------------------------------
\section{Opertatoren Überladung}
\begin{frame}{Opertatoren Überladung}
	\begin{itemize}
		\item Alle Typen können statische Methoden haben, welche Operatoren überschreiben und neu definieren können
		\item Nicht alle Operatoren können überschrieben werden
		\item Syntax:
		\begin{itemize}
			\item \texttt{public static \alert{<Datentyp>} operator \alert{<Operator>}\\( \alert{[Übergabeparameter]} ) \{ \alert{<Code>} \}}
		\end{itemize}
		\item Verschiedene Operatoren brauchen verschiedene Übergabeparameter und Rückgabeparameter
		\begin{itemize}
			 \item Man kann den selben Operator mit verschiedenen Übergabetypen überladen
		\end{itemize}
	\end{itemize}
\end{frame}

\begin{frame}{Opertatoren Überladung}
	\textbf{Operatoren}\\
	\begin{description}
		\item[unäre] \alert{\texttt{+}}, \alert{\texttt{-}}, \alert{\texttt{!}}, \alert{\texttt{\~}}, \alert{\texttt{++}}, \alert{\texttt{--}}, \alert{\texttt{true}}, \alert{\texttt{false}}
		\item[binär] \alert{\texttt{+}}, \alert{\texttt{-}}, \alert{\texttt{*}}, \alert{\texttt{/}}, \alert{\texttt{\%}}, \alert{\texttt{\&}}, \alert{\texttt{|}}, \alert{\texttt{\^}}, \alert{\texttt{<<}}, \alert{\texttt{>>}}
		\item[vergleich]  \alert{\texttt{==}}, \alert{\texttt{!=}}, \alert{\texttt{<}}, \alert{\texttt{>}}, \alert{\texttt{<=}}, \alert{\texttt{>=}}
	\end{description}
	\begin{itemize}
		\item Andere Operatoren können nicht überschrieben werden
		\item Operatoren wie \alert{+=} und \alert{\&\&} werden durch \alert{+} und \alert{\&} beschrieben
		\item \alert{true} und \alert{false} geben dem Objekt die Möglichkeit wie bool behandelt zu werden
		\item Bei Vergleichoperatoren muss der jeweilige Gegensatz auch überschrieben werden (z.B. bei \alert{==} auch \alert{!=} überschreiben)
	\end{itemize}
\end{frame}

\begin{frame}{Opertatoren Überladung}
	\lstinputlisting{resources/08_methoden_fortgeschritten_1/operation_overload_creation.cs}
	\lstinputlisting{resources/08_methoden_fortgeschritten_1/operation_overload_using.cs}
\end{frame}

% ----------------------- delegate ------------------------------
\section{delegate}
\begin{frame}{delegate}
	\begin{itemize}
		\item Auch Funktionspointer gennant
		\item Schlüsselwort zur Kapselung der Signatur von Methoden, um Referenzen auf diese abbilden zu können
		\item Es wird für diese (wie bei den Methoden) Übergabeparameter und Rückgabetyp deklariert
			\begin{itemize}
				\item Alle Methoden mit dieser Signatur können durch den Delegate dargestellt werden
			\end{itemize}
		\item Wird genutzt um eine Methode aufzurufen ohne auf dessen Objekt eine Referenz zuhaben
		\item Syntax (zur Deklarierung):
		\begin{itemize}
			\item \texttt{delegate \alert{<Rückgabetyp> <Delegatename>}\\( \alert{[Übergabeparameter]} );}
		\end{itemize}
		\item Syntax (zur Instanziierung):
		\begin{itemize}
			\item \texttt{\alert{<Delegantename> <Variablenname>} = \alert{<Methodenname>};}
		\end{itemize}
		\item Variable wird genauso genutzt wie eine normale Methode
	\end{itemize}
\end{frame}

\begin{frame}{delegate}
	\lstinputlisting{resources/08_methoden_fortgeschritten_1/delegate_creation.cs}
	\lstinputlisting{resources/08_methoden_fortgeschritten_1/delegate_using.cs}
\end{frame}

% ----------------------- Anonyme Methoden ------------------------------
\section{Anonyme Methoden}
\begin{frame}{Anonyme Methoden}
	\begin{itemize}
		\item Haben keinen eigenen Methodenkopf
		\item Können nur durch Delegate-Variablen aufgerufen und genutzt werden
		\item Es gibt 2 Arten der Erstellung:
		\begin{itemize}
			\item Durch Schlüsselwort \alert{delegate}
			\item Über Lamda-Terme
		\end{itemize}	
	\end{itemize}
\end{frame}

\subsection{delegate}
\begin{frame}{delegate}
	\begin{itemize}
		\item \alert{delegate} kann auch als Methodendefinition genutzt werden
		\item Syntax:
		\begin{itemize}
			\item \texttt{\alert{<Delegate-Variable>} = \\delegate( \alert{[Übergabeparameter]} ) \{ \alert{<Code>} \};}
		\end{itemize}
	\end{itemize}
	\lstinputlisting{resources/08_methoden_fortgeschritten_1/anonym_method_delegate_creation.cs}
	\lstinputlisting{resources/08_methoden_fortgeschritten_1/anonym_method_delegate_using.cs}	
\end{frame}

\subsection{Lamda}
\begin{frame}{Lamda}
	\begin{itemize}
		\item Benutzt den Lamda-Operator \alert{\texttt{=>}}
		\item Syntax:
		\begin{itemize}
			\item \texttt{( \alert{[Übergabeparameter]} ) => \{ \alert{<Code>} \};}
			\item Es gibt Unterschiede in der Syntax zwischen den einzelnen Lamda Arten
			\item Übergabeparameter brauchen nicht immer einen Typ, da dieser hergeleitet werden kann
		\end{itemize}
	\end{itemize}
\end{frame}

\subsubsection{Ausdruck-Lamdas}
\begin{frame}{Ausdruck-Lamdas}
	\begin{itemize}
		\item Besitzt im Körper nur einen Ausdruck und braucht somit keine geschweiften Klammern
		\item Gibt als Ergebnis das Ergebnis des einen Ausdrucks zurück
	\end{itemize}
	\lstinputlisting{resources/08_methoden_fortgeschritten_1/anonym_method_delegate_creation_2.cs}
	\lstinputlisting{resources/08_methoden_fortgeschritten_1/expression_lamda.cs}	
\end{frame}

\subsubsection{Anweisung-Lamdas}
\begin{frame}{Anweisung-Lamdas}
	\begin{itemize}
		\item Besitzen zur Eingrenzung des Körpers geschweifte Klammern
		\item Müssen nun (wenn kein \alert{void} als Rückgabetyp) ein \alert{return} besitzen	
	\end{itemize}
	\lstinputlisting{resources/08_methoden_fortgeschritten_1/anonym_method_delegate_creation_3.cs}
	\lstinputlisting{resources/08_methoden_fortgeschritten_1/statement_lamda.cs}
\end{frame}

\subsubsection{Variablenbereich für Lamdas}
\begin{frame}{Variablenbereich für Lamdas}
	\begin{itemize}
		\item Lamdas können auf Variablen außerhalb ihres eigenen Gültigkeitsbereichs/Scopes verweisen und diese nutzen
		\item Diese werden zur Verwendung im Lamda gespeichert
		\item In diesem Fall werden lokale Referenzen zu den Objekten im Lamda gespeichert
		\item So eine aüßere Variable muss definitiv vorher zugewiesen sein
	\end{itemize}
	\lstinputlisting{resources/08_methoden_fortgeschritten_1/anonym_method_delegate_creation_3.cs}
\end{frame}

\begin{frame}{Variablenbereich für Lamdas}
	\lstinputlisting{resources/08_methoden_fortgeschritten_1/variable_scope_lamda.cs}
\end{frame}

\end{document}
