% The Slide Definitions
\input{../templates/course_definitions}

% Author and Course information
% This Document contains the information about this course.

% Authors of the slides
\author{Marc Satkowski, Sascha Peukert}

% Name of the Course
\institute{C\texttt{\#} Kurs}

% Fancy Logo
\titlegraphic{\hfill\includegraphics[height=1.25cm]{../templates/fsr_logo_cropped}}


% Presentation title
\title{Collections, Generics und besondere Klassen}
\date{\today}


\begin{document}

\maketitle

\begin{frame}{Gliederung}
	\setbeamertemplate{section in toc}[sections numbered]
	\tableofcontents
\end{frame}

% ----------------------- Abstrakt ------------------------------
\section{Abstrakt}
\begin{frame}{Abstrakt}
	\begin{itemize}
		\item Schlüsselwort \alert{abstract}
		\begin{itemize}
			\item Kann für Klassen, Methoden und Eigenschaften genutzt werden		
		\end{itemize}
		\item Gibt an das keine Implementierung existiert
		\item Von Klassen die \alert{abstract} sind, können keine Objekte initialisiert werden
		\item Allerdings kann und muss von diesen geerbt werden um sie zu nutzen
		\item \alert{abstract} Methoden und Eigenschaften besitzen keinen Körper
		\begin{itemize}
			\item Dementsprechend wird eine Klasse, die von einer abstrakten Klasse erbt, gezwungen diese Methoden und Eigenschaften zu überschreiben und zu implementieren
			\item Eigenschaften können dennoch entscheiden ob sie ein \alert{get} oder \alert{set} brauchen und die abgeleitete Klasse darf \textbf{nur} die vorgegebenen implementieren
		\end{itemize}
	\end{itemize}
\end{frame}

\begin{frame}{Abstrakt}
	\lstinputlisting{resources/06_collections_generics_und_besondere_klassen/abstract.cs}
\end{frame}

% ----------------------- Interface ------------------------------
\section{Interface}
\begin{frame}{Interface}
	\begin{itemize}
		\item Zu dt. Schnittstelle
		\item Enthält nur die \alert{Signaturen} von Methoden und Eigenschaften
		\begin{itemize}
			\item Das heißt keine Implementierung
		\end{itemize}
		\item Interfaces und Klassen können von beliebig vielen Interfaces erben
		\item Lösung zur nicht vorhandenen Multivererbung
		\item Die Klassen, die von einem \alert{interface} erben, müssen alle in dem Interface definierten Methoden und Eigenschaften implementieren
		\item Wie bei der Klassenvererbung können Objekte einer Klasse die von einem Interface erbt auch als Typ des Interfaces interpretiert werden
	\end{itemize}
\end{frame}

\begin{frame}{Interface}
	\lstinputlisting{resources/06_collections_generics_und_besondere_klassen/interface.cs}
\end{frame}

\begin{frame}{Interface}
	\lstinputlisting{resources/06_collections_generics_und_besondere_klassen/interface_2.cs}
\end{frame}

% ----------------------- Generics ------------------------------
\section{Generics}
\begin{frame}{Generics}
	\begin{itemize}
		\item Auch Typparameter genannt
		\item Ermöglicht Typsicherheit ohne Angabe eines expliziten Types
		\item Syntax:
		\begin{itemize}
			\item \texttt{class \alert{<Klassenname>} <\alert{<Typparameter>}> \{ \}}
		\end{itemize}
	\end{itemize}
	\textbf{Einschränkung des Typparameters}\\
	\begin{itemize}
		\item Syntax:
		\begin{itemize}
			\item \texttt{class \alert{<Klassenname>} <\alert{<Typparameter>}> where \alert{<Typparameter>} : \alert{<Einschränkung>} \{ \}}
		\end{itemize}
		\item Einschränkungen können sein:
		\begin{itemize}
			\item Basisklasse, Interfaces
			\item \alert{struct} (Werttyp), \alert{class} (Referenztyp)
			\item \alert{new()} (Typparameter muss einen öffentlichen, leeren Konstruktor haben)
		\end{itemize}
	\end{itemize}
\end{frame}

\begin{frame}{Generics}
	\lstinputlisting{resources/06_collections_generics_und_besondere_klassen/generics_creation.cs}
	\lstinputlisting{resources/06_collections_generics_und_besondere_klassen/generics_using.cs}
\end{frame}

% ----------------------- Collections ------------------------------
\section{Collections}
\begin{frame}{Collections}
	\begin{itemize}
		\item Ermöglicht die Gruppierung von Objekte gleichen Typs
		\item Bieten verschiedene Datentypen wie verkettete Listen, Hash-Maps
		\item Bieten Methoden zum Sortieren, Durchsuchen und Bearbeiten
		\begin{itemize}
			\item Nutzen dafür Generics
		\end{itemize}
		\item Haben eine dynamische Größe, können somit während der Ausführung schrumpfen und wachsen
		\item Namespace:
		\begin{itemize}			
			\item \alert{System.Collection.Generic}
		\end{itemize}
	\end{itemize}
\end{frame}

\subsection{List}
\begin{frame}{List}
	\begin{itemize}
		\item Eine Liste beliebiger Elemente
		\item Elemente können über Index abgerufen werden (ähnlich zum Array)
	\end{itemize}
	\lstinputlisting{resources/06_collections_generics_und_besondere_klassen/list.cs}
\end{frame}

\subsection{Dictionary}
\begin{frame}{Dictionary}
	\begin{itemize}
		\item Ein "Lexikon" von Paaren der Form (Schlüssel, Wert)
		\item Jeder Schlüssel kann nur genau einmal vorkommen
		\item Kann über einen Schlüssel angesprochen werden aber nicht über einen Index, ähnlich zum Array
	\end{itemize}
	\lstinputlisting{resources/06_collections_generics_und_besondere_klassen/dictionary.cs}
\end{frame}

\subsection{Weitere}
\begin{frame}{Weitere}
	\textbf{Stack}\\
	\begin{itemize}
		\item Ein ''Stapel'' an Elementen
		\item Arbeitet nach dem LIFO-Prinzip
		\item \alert{.Push()} (Hinzufügen eines Elementes), \alert{.Pop()} (Entfernen eines Elementes)
	\end{itemize}
	\lstinputlisting{resources/06_collections_generics_und_besondere_klassen/stack.cs}
	\textbf{Queue}\\
	\begin{itemize}
		\item Eine ''Schlange'' an Elementen
		\item Arbeitet nach dem FIFO-Prinzip
		\item \alert{.Enqueue()} (Hinzufügen eines Elementes), \alert{.Dequeue()} (Entfernen eines Elementes)
	\end{itemize}
	\lstinputlisting{resources/06_collections_generics_und_besondere_klassen/queue.cs}
\end{frame}

\subsection{foreach}
\begin{frame}{foreach}
	\begin{itemize}
		\item Läuft durch die Elemente eines Arrays oder einer Collection bestimmten Typs durch 
		\item Weist bei jedem Durchlauf ein neues Element aus dem Array oder der Collection der Variable zu
		\begin{itemize}
			\item Der Typ stimmt mit dem Generic der Collection überein
			\item Bei Dictionarys ist der Typ ein \alert{KeyValuePair}		
		\end{itemize}
		\item Syntax:
		\begin{itemize}
			\item \texttt{foreach( \alert{<Datentyp> <Variablenname>} in \alert{<Aufzählung>} )\\ \{ \alert{<Code>} \}}
		\end{itemize}
	\end{itemize}
	\lstinputlisting{resources/06_collections_generics_und_besondere_klassen/foreach.cs}	
\end{frame}

\end{document}
