% The Slide Definitions
\input{../templates/course_definitions}

% Author and Course information
% This Document contains the information about this course.

% Authors of the slides
\author{Marc Satkowski, Sascha Peukert}

% Name of the Course
\institute{C\texttt{\#} Kurs}

% Fancy Logo
\titlegraphic{\hfill\includegraphics[height=1.25cm]{../templates/fsr_logo_cropped}}


% Presentation title
\title{Vererbung und Polymorphie}
\date{\today}


\begin{document}

\maketitle

\begin{frame}{Gliederung}
	\setbeamertemplate{section in toc}[sections numbered]
	\tableofcontents
\end{frame}

% ----------------------- Abstrakt ------------------------------
\section{Abstrakt}
\begin{frame}{Abstrakt}
	\begin{itemize}
		\item Schlüsselwort \alert{abstract}
		\begin{itemize}
			\item Kann für Klassen, Methoden und Eigenschaften genutzt werden		
		\end{itemize}
		\item Gibt an ob keine Implementierung existieren darf
		\item Von Klassen die \alert{abstract} sind, können keine Objekte initialisiert werden
		\begin{itemize}
			\item Dementsprechend muss von diesen geerbt werden um sie zu nutzen
		\end{itemize}
		\item Methoden und Eigenschaften besitzen keinen Körper wenn diese \alert{abstract} sind
		\begin{itemize}
			\item Dementsprechend wird eine Klasse, die davon erbt, gezwungen diese Methoden und Eigenschaften zu überschreiben und zu implementieren
			\item Eigenschaften können dennoch entscheiden ob sie ein \alert{get} oder \alert{set} brauchen und die abgeleitete Klasse darf nur die vorgegebenen implementieren
		\end{itemize}
	\end{itemize}
\end{frame}

\begin{frame}{Abstrakt}
	\lstinputlisting{resources/06_collections_generics_und_besondere_klassen/abstract.cs}
\end{frame}

% ----------------------- Interface ------------------------------
\section{Interface}
\begin{frame}{Interface}
	\begin{itemize}
		\item Zu dt. Schnittstelle
		\item Enthält nur die \alert{Signaturen} von Methoden und Eigenschaften
		\begin{itemize}
			\item Das heißt diese haben alle keinen Körper
		\end{itemize}
		\item Interfaces und Klassen können von beliebig vielen Interfaces erben
		\item Die Klassen, die von Interfaces erben, müssen alle in den Interfaces definierten Methoden und Eigenschaften implementieren
	\end{itemize}
\end{frame}

\begin{frame}{Interface}
	\lstinputlisting{resources/06_collections_generics_und_besondere_klassen/interface.cs}
\end{frame}

% ----------------------- Generics ------------------------------
\section{Generics}
\begin{frame}{Generics}
	\begin{itemize}
		\item Auch Typparameter genannt
		\item Ermöglicht Typsicherheit ohne Angabe eines expliziten Types
		\item Syntax:
		\begin{itemize}
			\item \texttt{class \alert{<Klassenname>} <\alert{<Typparameter>}> \{ \}}
		\end{itemize}
	\end{itemize}
	\textbf{Einschränkungen}\\
	\begin{itemize}
		\item Man kann den Typparameter auch einschränken
		\item Syntax:
		\begin{itemize}
			\item \texttt{\alert{<Klassenkopf>} where \alert{<Typparameter>} : \alert{<Einschränkung>}\\ \{ \}}
		\end{itemize}
		\item Einschränkungen können sein:
		\begin{itemize}
			\item Basisklasse, Interfaces
			\item \alert{struct} (Werttyp), \alert{class} (Referenztyp)
			\item \alert{new()} (Typparameter muss einen öffentlichen, leeren Konstruktor haben)
		\end{itemize}
	\end{itemize}
\end{frame}

\begin{frame}{Generics}
	\lstinputlisting{resources/06_collections_generics_und_besondere_klassen/generics_creation.cs}
	\lstinputlisting{resources/06_collections_generics_und_besondere_klassen/generics_using.cs}
\end{frame}

% ----------------------- Collections ------------------------------
\section{Collections}
\begin{frame}{Collections}
	\begin{itemize}
		\item Ermöglicht die Gruppierung von Objekte gleichen Types
		\item Bieten verschiedene Datentypen wie verkettete Listen, Hash-Maps, ...
		\item Bieten Methoden zum Sortieren, Durchsuchen und Bearbeiten
		\begin{itemize}
			\item Nutzen dafür Generics
		\end{itemize}
		\item Haben eine dynamische Größe, können somit während der Ausführung schrumpfen und wachsen
		\item Namespace:
		\begin{itemize}			
			\item \alert{System.Collection.Generic}
		\end{itemize}
	\end{itemize}
\end{frame}

\subsection{List}
\begin{frame}{List}
	\begin{itemize}
		\item Eine Liste beliebiger Elemente
		\item Kann über einen Index angesteuert werden (ähnlich zum Array)
	\end{itemize}
	\lstinputlisting{resources/06_collections_generics_und_besondere_klassen/list.cs}
\end{frame}

\subsection{Dictionary}
\begin{frame}{Dictionary}
	\begin{itemize}
		\item Ein "Lexikon" von Paaren mit beliebigen Schlüssel und Wert
		\item Jeder Schlüssel kann nur genau einmal vorkommen
		\item Kann ebenfalls, ähnlich zum Array, angesprochen werden, aber dieses mal wird ein Schlüssel übergeben
	\end{itemize}
	\lstinputlisting{resources/06_collections_generics_und_besondere_klassen/dictionary.cs}
\end{frame}

\subsection{Weitere}
\begin{frame}{Weitere}
	\textbf{Stack}\\
	\begin{itemize}
		\item Ein ''Stapel'' an Elementen
		\item Arbeitet nach dem LIFO-Prinzip
		\item \alert{.Push()} (Hinzufügen eines Elementes), \alert{.Pop()} (Entfernen eines Elementes)
	\end{itemize}
	\lstinputlisting{resources/06_collections_generics_und_besondere_klassen/stack.cs}
	\textbf{Queue}\\
	\begin{itemize}
		\item Eine ''Schlange'' an Elementen
		\item Arbeitet nach dem FIFO-Prinzip
		\item \alert{.Enqueue()} (Hinzufügen eines Elementes), \alert{.Dequeue()} (Entfernen eines Elementes)
	\end{itemize}
	\lstinputlisting{resources/06_collections_generics_und_besondere_klassen/queue.cs}
\end{frame}

\subsection{foreach}
\begin{frame}{foreach}
	\begin{itemize}
		\item Läuft durch ein/e Array/Collection an Elementen eines Typen durch 
		\item Weist jeden Durchlauf ein neues Element aus dem Array/der Collection der Variable zu
		\begin{itemize}
			\item Der Typ stimmt mit dem Generic der Collection überein
			\item Bei Dictionarys ist der Typ ein \alert{KeyValuePair}		
		\end{itemize}
		\item Syntax:
		\begin{itemize}
			\item \texttt{foreach( \alert{<Datentyp> <Variablenname>} in \alert{<Aufzählung>} )\\ \{ \alert{<Code>} \}}
		\end{itemize}
	\end{itemize}
	\lstinputlisting{resources/06_collections_generics_und_besondere_klassen/foreach.cs}	
\end{frame}

\end{document}
