% The Slide Definitions
\input{../templates/course_definitions}

% Author and Course information
% This Document contains the information about this course.

% Authors of the slides
\author{Marc Satkowski, Sascha Peukert}

% Name of the Course
\institute{C\texttt{\#} Kurs}

% Fancy Logo
\titlegraphic{\hfill\includegraphics[height=1.25cm]{../templates/fsr_logo_cropped}}


% Presentation title
\title{Vererbung und Polymorphie}
\date{\today}


\begin{document}

\maketitle

\begin{frame}{Gliederung}
	\setbeamertemplate{section in toc}[sections numbered]
	\tableofcontents
\end{frame}

% ----------------------- Methodenüberladung ------------------------------
\section{Methodenüberladung}
\begin{frame}{Methodenüberladung}
	\begin{itemize}
		\item Mehrere Methoden können den selben Namen haben
		\item Diese Unterscheiden sich dann in dem Rückgabewert und/oder den Übergabeparametern (Reihenfolge beachten!) 
		\item Wird genutzt wenn es mehrere Wege gibt eine Berechnung/Algorithmus auszuführen
	\end{itemize}
	\lstinputlisting{resources/04_vererbung_und_polymorphie/method_overload.cs}
\end{frame}

% ----------------------- Vererbung ------------------------------
\section{Vererbung}
\begin{frame}{Vererbung}
	\begin{itemize}
		\item Ermöglicht es einer Klasse definiertes Verhalten einer anderen Klasse (Methoden, Variablen, Eigenschaften) zu
		\begin{itemize}
			\item überschreiben
			\item erweitern
			\item benutzen
		\end{itemize}
		\item Alle Klassen erben von der Klasse \alert{Object}
		\item Klassen können maximal von einer Klasse erben
		\item Die Klasse die Verhalten vererbt heiß \alert{Basisklasse} und die Klasse, die es erbt, heißt \alert{abgeleitete Klasse}
		\item Syntax:
		\begin{itemize}
			\item \texttt{class \alert{<Klassenname>} : \alert{<Basisklasse>}}
		\end{itemize}
	\end{itemize}
	\lstinputlisting{resources/04_vererbung_und_polymorphie/inheritance.cs}
\end{frame}

\subsection{Polymorphie}
\begin{frame}{Polymorphie}
	\begin{itemize}
		\item Zu dt. Vielgestaltigkeit
		\item Bedeutet das abgeleitete Klassen zur Laufzeit als ihre Basisklassen behandelt werden können
		\begin{itemize}
			\item Somit stimmt der Typ des Objekts nicht mehr mit dem Laufzeittyp überein
		\end{itemize}
		\item Basisklassen können Methoden und Eigenschaften besitzen, die von den abgeleiteten Klassen überschrieben werden können
		\item Während der Nutzung eines Objekts als Basisklasse wird der Laufzeittyp ermittelt und die überschrieben/veränderten Methoden und Eigenschaften genutzt
		\item Falls es keine Überschreibungen gibt, wir die Methode aus der Basisklasse genutzt
	\end{itemize} 
\end{frame}

\subsection{Methoden- \& Eigenschaftsüberschreibung}
\begin{frame}{Methoden- \& Eigenschaftsüberschreibung}	
	\textbf{virtual}\\
	\begin{itemize}
		\item Erlaubt das Überschreiben einer Methode
		\item Wird an die Methoden einer Basisklasse geschrieben
	\end{itemize}
	\lstinputlisting{resources/04_vererbung_und_polymorphie/virtual.cs}
\end{frame}

\begin{frame}{Methoden- \& Eigenschaftsüberschreibung}	
	\textbf{override}\\
	\begin{itemize}
		\item Gibt an, dass es sich um eine Überschreibung einer Methode aus der Basisklasse handelt
		\begin{itemize}
			\item Es darf bloß überschrieben werden, wenn die Methode in der Basisklasse ein \alert{virtual} hat
		\end{itemize}
		\item Wird an die Methoden in der abgeleiteten Klasse geschrieben
	\end{itemize}
	\textbf{new}\\
	\begin{itemize}
		\item Gibt an ob eine bekannte Methode aus der Basisklasse überschrieben werden soll, obwohl diese es nicht erlaubt
		\begin{itemize}
			\item Diese Methode hat also kein \alert{virtual} in der Basisklasse
		\end{itemize}
	\end{itemize}
\end{frame}

\begin{frame}{Methoden- \& Eigenschaftsüberschreibung}
	\lstinputlisting{resources/04_vererbung_und_polymorphie/override_new.cs}	
	\textbf{Hinweis:}
	\begin{itemize}
		\item Die Überschreibung von Eigenschaften geht nur, wenn diese schon einen Körper zu ihrem \alert{get} und/oder \alert{set} haben
	\end{itemize}
\end{frame}

\subsection{Weitere Schlüsselwörter}
\begin{frame}{Weitere Schlüsselwörter}
	\textbf{sealed}\\
	\begin{itemize}
		\item Modifikator für Klassen und Methoden
		\item Verhindert das diese Klasse als Basisklasse genutzt werden kann
		\item Bei Methoden wird verhindert, dass diese weiter überschrieben werden können
	\end{itemize}
	\lstinputlisting{resources/04_vererbung_und_polymorphie/sealed.cs}
\end{frame}

\begin{frame}{Weitere Schlüsselwörter}
	\textbf{as}\\
	\begin{itemize}
		\item Wird als Casting-Operator zwischen vererbten Klassen genutzt
		\item Syntax:
		\begin{itemize}
			\item \texttt{(\alert{<Variable>} as \alert{<Datentyp/Klasse>})}
		\end{itemize}
		\item Nach der Klammer kann auf die Variable als der neue Typ darauf zugegriffen werden
		\item Falls die Umwandlung nicht möglich ist, wird die Operation als \alert{null} ausgewertet
	\end{itemize}	
	\lstinputlisting{resources/04_vererbung_und_polymorphie/as.cs}
\end{frame}

\begin{frame}{Weitere Schlüsselwörter}
	\textbf{base}\\
	\begin{itemize}
		\item Es ist das Objekt der Basisklasse
		\item Ermöglicht es auf die Methoden- und Eigenschaftsimplementationen der Basisklasse zuzugreifen
		\item Dies geht auch, wenn man diese in seiner Klasse schon überschrieben hat
	\end{itemize}
	\begin{itemize}
		\item Muss genutzt werden, wenn die Basisklasse einen Konstruktor mit Übergabeparametern hat
		\begin{itemize}
			\item Dabei wird der Konstruktor der Basisklasse vor dem eigenen aufgerufen
			\item Es müssen auch die "Bedingungen" des Basisklasse erfüllt sein
		\end{itemize}
		\item Syntax:
		\begin{itemize}
			\item \texttt{\alert{Klasse}(\alert{[Übergabeparameter-B][Übergabeparameter]})\\ : base(\alert{[Übergabeparameter-B]}) \{ \alert{<Code>} \}}
		\end{itemize}
	\end{itemize}
\end{frame}

\begin{frame}{Weitere Schlüsselwörter}
	\lstinputlisting{resources/04_vererbung_und_polymorphie/base.cs}
\end{frame}

% ----------------------- Abstrakt ------------------------------
\section{Abstrakt}
\begin{frame}{Abstrakt}
	\begin{itemize}
		\item Schlüsselwort \alert{abstract}
		\begin{itemize}
			\item Kann für Klassen, Methoden und Eigenschaften genutzt werden		
		\end{itemize}
		\item Gibt an ob eine Implementierung existiert
		\item Von Klassen die \alert{abstract} sind, können keine Objekte initialisiert werden
		\begin{itemize}
			\item Dementsprechend muss von diesen geerbt werden um diese zu Nutzen
		\end{itemize}
		\item Methoden und Eigenschaften besitzen keinen Körper wenn diese \alert{abstract} sind
		\begin{itemize}
			\item Dementsprechend wird eine Klasse, die davon erbt, gezwungen diese Methoden und Eigenschaften zu überschreiben und zu implementieren
		\end{itemize}
	\end{itemize}
\end{frame}

\begin{frame}{Abstrakt}
	\lstinputlisting{resources/04_vererbung_und_polymorphie/abstract.cs}
\end{frame}

% ----------------------- Interface ------------------------------
\section{Interface}
\begin{frame}{Interface}
	\begin{itemize}
		\item Zu dt. Schnittstelle
		\item Enthält nur die \alert{Signaturen} von Methoden und Eigenschaften
		\begin{itemize}
			\item Das heißt diese haben alle keinen Körper
		\end{itemize}
		\item Interfaces und Klassen können von beliebig vielen Interfaces erben
		\item Die Klassen, die von Interfaces erben, werden gezwungen zu allen Signaturen Implementationen zu haben
	\end{itemize}
\end{frame}

\begin{frame}{Interface}
	\lstinputlisting{resources/04_vererbung_und_polymorphie/interface.cs}
\end{frame}

\end{document}
