% The Slide Definitions
\input{../templates/course_definitions}

% Author and Course information
% This Document contains the information about this course.

% Authors of the slides
\author{Marc Satkowski, Sascha Peukert}

% Name of the Course
\institute{C\texttt{\#} Kurs}

% Fancy Logo
\titlegraphic{\hfill\includegraphics[height=1.25cm]{../templates/fsr_logo_cropped}}


% Presentation title
\title{Vererbung und Polymorphie}
\date{\today}


\begin{document}

\maketitle

\begin{frame}{Gliederung}
	\setbeamertemplate{section in toc}[sections numbered]
	\tableofcontents
\end{frame}

% ----------------------- Methodenüberladung ------------------------------
\section{Methodenüberladung}
\begin{frame}{Methodenüberladung}
	\begin{itemize}
		\item Mehrere Methoden können den selben Namen haben
		\item Diese Unterscheiden sich dann in dem Rückgabewert und/oder den Übergabeparametern
		\item Wird genutzt wenn es mehrere Wege gibt eine Berechnung/Algorithmus auszuführen
	\end{itemize}
	\lstinputlisting{resources/04_vererbung_und_polymorphie/method_overload.cs}
\end{frame}

% ----------------------- Vererbung ------------------------------
\section{Vererbung}
\begin{frame}{Vererbung}
	\begin{itemize}
		\item Ermöglicht es einer Klasse definiertes Verhalten einer anderen Klasse (Methoden, Variablen, Eigenschaften) zu
		\begin{itemize}
			\item überschreiben
			\item erweitern
			\item ändern
			\item benutzen
		\end{itemize}
		\item Alle Klassen erben von der Klasse \alert{Object}
		\item Die Klasse die Verhalten vererbt heiß \alert{Basisklasse} und die Klasse, die es erbt, heißt \alert{abgeleitete Klasse}
		\item Syntax:
		\begin{itemize}
			\item \texttt{class \alert{<Klassenname>} : \alert{<Basisklasse>}}
		\end{itemize}
	\end{itemize}
	\lstinputlisting{resources/04_vererbung_und_polymorphie/inheritance.cs}
\end{frame}

\subsection{Polymorphie}
\begin{frame}{Vererbung}
	% protected, base, as
\end{frame}

\subsection{Weitere Schlüsselwörter}
\begin{frame}{Weitere Schlüsselwörter}
	% virtual, override, sealdes, new
\end{frame}

% ----------------------- Abstrakt ------------------------------
\section{Abstrakt}
\begin{frame}{Abstrakt}

\end{frame}

% ----------------------- Interface ------------------------------
\section{Interface}
\begin{frame}{Interface}

\end{frame}

\end{document}
