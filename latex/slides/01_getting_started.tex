% The Slide Definitions
\input{../templates/course_definitions}

% Author and Course information
% This Document contains the information about this course.

% Authors of the slides
\author{Marc Satkowski, Sascha Peukert}

% Name of the Course
\institute{C\texttt{\#} Kurs}

% Fancy Logo
\titlegraphic{\hfill\includegraphics[height=1.25cm]{../templates/fsr_logo_cropped}}


% Presentation title
\title{Grundlagen von C\texttt{\#}}
\date{\today}


\begin{document}

\maketitle

\begin{frame}{Gliederung}
	\setbeamertemplate{section in toc}[sections numbered]
	\tableofcontents
\end{frame}

% ----------------------- Über den Kurs ------------------------------
\section{Über diesen Kurs}
\begin{frame}{Über diesen Kurs}
	\begin{itemize}
		\item 10 Kurseinheiten
		\item keine besonderen Programmierkenntnisse vorausgesetzt, aber nützlich
		\item Ressourcen
		\begin{itemize}
			\item Google (C\texttt{\#} meine Frage hier); man landet oft auf der Microsoft Seite
			\item \href{http://stackoverflow.com/}{Stackoverflow}
			\item unsere \href{https://github.com/fsr}{Github-Organisation}
		\end{itemize}
	\end{itemize}
\end{frame}

% ----------------------- Software ------------------------------
\section{Benötigte Software}
\begin{frame}{Benötigte Software}
	.Net-Framework
	\begin{itemize}
		\item Kann mit Visual Studio installiert werden
		\item oder mit \href{http://www.mono-project.com/}{Mono} für Linux, Windows oder Mac
	\end{itemize}
	IDEs
	\begin{itemize}
		\item \href{https://www.visualstudio.com/}{Visual Studio} für Windows
		\item \href{http://www.monodevelop.com/}{Mono Develop} für Linux, Windows oder Mac
		\item \href{https://www.visualstudio.com/de-de/products/code-vs.aspx}{Visual Code} für Linux, Windows oder Mac
		\begin{itemize}
			\item Benötigt noch eine C\texttt{\#}-Extension (Ctrl+P \texttt{->} ext install C\texttt{\#})
		\end{itemize}
	\end{itemize}
\end{frame}


% ----------------------- Hello World ------------------------------
\section{Hello World}
\begin{frame}{Hello World}
    \lstinputlisting{resources/01_getting_started/helloworld.cs}
\end{frame}

% ----------------------- Grundlagen der Sprache ------------------------------
\section{Grundlagen der Sprache}
\subsection{Wichtige Eigenschaften}
\begin{frame}{Wichtige Eigenschaften}
	\begin{itemize}
		\item Am Ende jedem Befehls steht ein Semikolon (;)
		\begin{itemize}
			\item Befehle können auch über mehrere Zeilen gezogen werden
		\end{itemize}
		\item Codeblöcke werden durch geschweifte Blöcke gekennzeichnet
		\item Einrückungen sind nicht zwingend aber hilfreich
		\item Methodendefinition
		\begin{itemize}
			\item \alert{\texttt{<Rückgabetyp><Methodenname>([Übergabeparameter])}}
		\end{itemize}
	\end{itemize}
\end{frame}

\subsection{Kommentare}
\begin{frame}{Kommentare}
	\begin{itemize}
		\item Werden vom Compiler nicht beachtet und dienen nur zum besseren Verständnis des Codes
		\item Außerdem kann aus Dokumentationskommentare eine Dokumentation erstellt werden
	\end{itemize}
	\lstinputlisting{resources/01_getting_started/comments.cs}
\end{frame}

\subsection{Datentypen}
\begin{frame}{Datentypen}
	\begin{tabular}{c|l}
		Name & Funktion \\ \hline
		\texttt{int, long} & Ganzzahl "beliebiger" Größe \\
		\texttt{uint, ulong} & Natürliche Zahle "beliebiger" Größe \\
		\texttt{float, double} & Gleitkommazahl "beliebiger" Größe \\
		\texttt{ufloat, udouble} & Positive Gleitkommazahl "beliebiger" Größe \\
        \texttt{decimal} & Präziser Bruch/Ganzzahl mit 29 signifikanten Stellen \\
		\texttt{bool} & Wahrheitswert (\texttt{True}, \texttt{False})\\
        \texttt{char} & Ein einzelnes Zeichen \\
		\texttt{string} & Eine Folge von Zeichen \\
	\end{tabular}
	\begin{itemize}
		 \item Die Variablen mit u (unsigned) als ersten Buchstaben werden normalerweise nicht genutzt
	\end{itemize}
\end{frame}

\subsection{Operatoren}
\begin{frame}{Operatoren}
	\begin{description}
	    \item[mathematisch] \alert{\texttt{+}}, \alert{\texttt{-}}, \alert{\texttt{*}}, \alert{\texttt{/}}, \alert{\texttt{\%}}
	    \item[vergleichend] \alert{\texttt{<}}, \alert{\texttt{>}}, \alert{\texttt{<=}}, \alert{\texttt{>=}}, \alert{\texttt{==}} (Gleichheit), \alert{\texttt{!=}} (Ungleichheit)
	    \item[logisch] \alert{\texttt{\&\&}} (Und), \alert{\texttt{||}} (Oder), \alert{\texttt{!}} (Negation)
	    \item[bitweise] \alert{\texttt{\&}}, \alert{\texttt{|}}, \alert{\texttt{<<}}, \alert{\texttt{>>}}, \alert{\texttt{\^}} (xor), \alert{\texttt{\~}} (invertieren)
	\end{description}
\end{frame}

\end{document}
