% The Slide Definitions
\input{../templates/course_definitions}

% Author and Course information
% This Document contains the information about this course.

% Authors of the slides
\author{Marc Satkowski, Sascha Peukert}

% Name of the Course
\institute{C\texttt{\#} Kurs}

% Fancy Logo
\titlegraphic{\hfill\includegraphics[height=1.25cm]{../templates/fsr_logo_cropped}}


% Presentation title
\title{Datenstrukturen}
\date{\today}


\begin{document}

\maketitle

\begin{frame}{Gliederung}
	\setbeamertemplate{section in toc}[sections numbered]
	\tableofcontents
\end{frame}

% ----------------------- Enummeration ------------------------------
\section{Enumeration}
\begin{frame}{Enumeration}
	\begin{itemize}
		\item Zu dt. Aufzählung
		\item Das Schlüsselwort dafür ist \alert{enum}
		\item Ist ein eigen erstellter Datentyp mit fester Anzahl konstanter Werte
		\item Den Werten werden Intern ein \alert{int}-Wert zugeordnet
		\begin{itemize}
			\item Diese beginnt Standardmäßig bei 0 und wird mit jedem neuen Element um 1 erhöht
			\item Diese Ordnung kann man auch selbst mit einem \alert{=} festlegen
			\item Man kann damit ein \alert{enum}-Wert in einen \alert{int}-Wert casten			
		\end{itemize}
		\item Syntax zur Nutzung:
		\begin{itemize}
			\item \texttt{\alert{<Enum> <Variablenname>} = \alert{<Enum>}.\alert{<Enumwert>};} oder
			\item \texttt{\alert{<Enum> <Variablenname>} = \alert{<Int-Wert>};}
		\end{itemize}
	\end{itemize}
\end{frame}

\begin{frame}{Enumeration}
	\lstinputlisting{resources/06_datenstrukturen/enum_creation.cs}
	\lstinputlisting{resources/06_datenstrukturen/enum_using.cs}
\end{frame}

% ----------------------- Structs ------------------------------
\section{Structs}
\begin{frame}{Structs}

\end{frame}

% ----------------------- Generics ------------------------------
\section{Generics}
\begin{frame}{Generics}

\end{frame}

% ----------------------- Collections ------------------------------
\section{Collections}
\begin{frame}{Collections}
	% List, Dictionary
\end{frame}

% ----------------------- Indexer ------------------------------
\section{Indexer}
\begin{frame}{Indexer}

\end{frame}

\end{document}
