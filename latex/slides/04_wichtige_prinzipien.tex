% The Slide Definitions
\input{../templates/course_definitions}

% Author and Course information
% This Document contains the information about this course.

% Authors of the slides
\author{Marc Satkowski, Sascha Peukert}

% Name of the Course
\institute{C\texttt{\#} Kurs}

% Fancy Logo
\titlegraphic{\hfill\includegraphics[height=1.25cm]{../templates/fsr_logo_cropped}}


% Presentation title
\title{Objektorientierung}
\date{\today}


\begin{document}

\maketitle

\begin{frame}{Gliederung}
	\setbeamertemplate{section in toc}[sections numbered]
	\tableofcontents
\end{frame}

% ----------------------- Zugriffsmodifikatoren/Sichtbarkeiten ------------------------------
\section{Zugriffsmodifikatoren/Sichtbarkeiten}
\begin{frame}{Zugriffsmodifikatoren/Sichtbarkeiten}
	\begin{itemize}
		\item Geben an, ob der zugehörige Code von anderen Code-Abschnitten verwendet werden kann
		\item Es gibt mehrere Arten
		\begin{description}
			\item[public] kann von überall zugriffen werden, solange darauf verweist wird
			\item[private] kann nur von Code in der eigenen Klasse benutzt werden
			\item[protected] kann nur von Code in der eigenen und von abgeleiteten Klassen benutzt werden
			\item[internal] kann von überall aus der eigenen Assembly heraus benutzt werden
		\end{description}
	\end{itemize}
\end{frame}

\begin{frame}{Zugriffsmodifikatoren/Sichtbarkeiten}
	\lstinputlisting{resources/04_wichtige_prinzipien/access_modifier.cs}
\end{frame}

% ----------------------- namespace und using ------------------------------
\section{namespace \& using}
\begin{frame}{namespace \& using}
	\textbf{namespace}\\
	\begin{itemize}
		\item Werden zur Organisation von Klassen und anderem Code genutzt 
		\item Können beliebig verschachtelt werden
		\item der volle Name einer Klasse beinhaltet die Namen aller Namespaces, in denen die Klasse liegt
		\item Vermeiden von Mehrdeutigen Bezeichnen 
		\begin{itemize}
			\item Zwei Klassen dürfen den selben Namen haben, wenn diese in unterschiedlichen Namespaces liegen		
		\end{itemize}
	\end{itemize}	
	\lstinputlisting{resources/04_wichtige_prinzipien/namespace.cs}
\end{frame}

\begin{frame}{namespace \& using}
	\textbf{using}\\
	\begin{itemize}
		\item Ein Schlüsselwort zur Nutzung eines bestimmten \alert{namespace}
		\item Dies erspart die Eingabe des gesamten namespace bei Nutzung einer Klasse
		\item Kann zu merhdeutigen Code führen, wenn 2 Namespaces eingebunden werden, die Elemente mit selben Namen beinhalten
	\end{itemize}
	\lstinputlisting{resources/04_wichtige_prinzipien/using.cs}
\end{frame}

% ----------------------- Methodenüberladung ------------------------------
\section{Methodenüberladung}
\begin{frame}{Methodenüberladung}
	\begin{itemize}
		\item Mehrere Methoden können den selben Namen haben
		\item Diese Unterscheiden sich dann in dem Rückgabewert und/oder den Übergabeparametern (Reihenfolge beachten!) 
		\item Wird genutzt wenn es mehrere Wege gibt eine Berechnung/Algorithmus auszuführen
	\end{itemize}
	\lstinputlisting{resources/04_wichtige_prinzipien/method_overload.cs}
\end{frame}

% ----------------------- null ------------------------------
\section{null}
\begin{frame}{null}
	\begin{itemize}
		\item Schlüsselwort für eine Referenz die auf kein Objekt zeigt
		\item Alle Referenztyp-Variablen können \alert{null} sein
		\item Versucht man ein \alert{null} Objekt zu dereferenzieren, wird eine NullReferenceException geworfen 
		\begin{itemize}
			\item Es ist möglich ein Objekt auf eine \alert{null} Referenz zu prüfen
		\end{itemize}
	\end{itemize}
	\lstinputlisting{resources/04_wichtige_prinzipien/null.cs}	
\end{frame}

% ----------------------- Wert- und Referenztyp ------------------------------
\section{Wert- \& Referenztyp}

\end{document}
