% The Slide Definitions
\input{../templates/course_definitions}

% Author and Course information
% This Document contains the information about this course.

% Authors of the slides
\author{Marc Satkowski, Sascha Peukert}

% Name of the Course
\institute{C\texttt{\#} Kurs}

% Fancy Logo
\titlegraphic{\hfill\includegraphics[height=1.25cm]{../templates/fsr_logo_cropped}}


% Presentation title
\title{Wichtige Prinzipien von C\texttt{\#}}
\date{\today}


\begin{document}

\maketitle

\begin{frame}{Gliederung}
	\setbeamertemplate{section in toc}[sections numbered]
	\tableofcontents
\end{frame}


% ----------------------- Dokumentationskommentare ------------------------------
\section{Dokumentationskommentare}
\begin{frame}{Dokumentationskommentare}
	\begin{itemize}
		\item Das Doku-Kommentar beginnt mit \texttt{\alert{///}} (3), anstatt \texttt{\alert{//}} (2)
		\item Stehen nur Klassen, Felder/Eigenschaften und Methoden
		\begin{itemize}
			\item Das Kommentar steht dabei immer direkt über den zu kommentierenden Code
		\end{itemize}
		\item Man kann aus diesen eine Dokumentation generieren
		\item Außerdem können einige IDEs (z.B. Visual Studio) daraus eine Kurzbeschreibung der Elemente während des Programmierens schaffen		
	\end{itemize}
\end{frame}

\begin{frame}{Dokumentationskommentare}
	\textbf{Tags}
	\begin{itemize}
		\item Sind Teil eines Dokumentationskommentars
		\item Geben an, was genau von dem Code dokumentiert werden soll
		\item 3 wichtige Tags:
		\begin{description}
			\item[summary] Beschreibung des zu kommentierenden Codes\\ Für alle 3 Arten
			\item[param] Beschreibung der Übergabeparameter einer Methode
			\item[returns] Beschreibung des Rückgabewertes einer Methode
		\end{description}
		\item \href{https://msdn.microsoft.com/de-de/library/5ast78ax.aspx}{\alert{Hier}} findet man weitere Tags
	\end{itemize}
\end{frame}

\begin{frame}{Dokumentationskommentare}
	\begin{itemize}
		\item Die genannten Tags am Beispiel einer Methode:
	\end{itemize}
	\lstinputlisting{resources/04_wichtige_prinzipien/documentation.cs}	
\end{frame}

\begin{frame}
	\begin{itemize}
		\item Und nun mit Inhalt:
	\end{itemize}
	\lstinputlisting{resources/04_wichtige_prinzipien/documentation_example.cs}		
\end{frame}

% ----------------------- Scope ------------------------------
\section{Gültigkeit (Scope)}
\begin{frame}{Gültigkeit (Scope)}
	\begin{itemize}
		\item Variablen, Member, Klassen und andere Elementen sind nur innerhalb eines bestimmten Bereichs gültig:
		\begin{itemize}
			\item Methoden und Felder nur in deren Klassen
			\item Parameter nur in deren Methoden
			\item Lokale Variablen innerhalb ihrer Blöcke
		\end{itemize}
		\item Dieser Bereich ist definiert durch die Position an denen sie deklariert wurden
	\end{itemize}
\end{frame}

\begin{frame}{Gültigkeit (Scope)}
	\lstinputlisting{resources/04_wichtige_prinzipien/scope_1.cs}
\end{frame}

\begin{frame}{Gültigkeit (Scope)}
	\lstinputlisting{resources/04_wichtige_prinzipien/scope_2.cs}
\end{frame}

% ----------------------- namespace und using ------------------------------
\section{namespace \& using}
\begin{frame}{namespace \& using}
	\textbf{namespace}\\
	\begin{itemize}
		\item Werden zur Organisation von Klassen und anderem Code genutzt 
		\item Können beliebig verschachtelt werden
		\item der volle Name einer Klasse beinhaltet die Namen aller Namespaces, in denen die Klasse liegt
		\begin{itemize}
			\item es gibt eine Aussnahme
		\end{itemize}
		\item Vermeiden von mehrdeutigen Bezeichnern 
		\begin{itemize}
			\item Zwei Klassen dürfen den selben Namen haben, wenn diese in unterschiedlichen Namespaces liegen		
		\end{itemize}
	\end{itemize}	
	\lstinputlisting{resources/04_wichtige_prinzipien/namespace.cs}
\end{frame}

\begin{frame}{namespace \& using}
	\textbf{using}\\
	\begin{itemize}
		\item Ein Schlüsselwort zur Nutzung eines bestimmten \alert{namespace}
		\item Dies erspart die Eingabe des gesamten Namespace bei Nutzung einer Klasse
		\item Kann zu mehrdeutigen Code führen, wenn 2 Namespaces eingebunden werden, die Elemente mit selben Namen beinhalten
	\end{itemize}
	\begin{itemize}
		\item \alert{using} kann aber auch in anderen Zusammenhängen genutzt werden!
	\end{itemize}
\end{frame}

\begin{frame}	
	\lstinputlisting{resources/04_wichtige_prinzipien/using.cs}
\end{frame}

% ----------------------- Methodenüberladung ------------------------------
\section{Methodenüberladung}
\begin{frame}{Methodenüberladung}
	\begin{itemize}
		\item Mehrere Methoden können den selben Namen haben
		\item Diese Unterscheiden sich dann in dem Rückgabewert und/oder den Übergabeparametern
		\item Wird genutzt wenn es mehrere Wege gibt eine Berechnung/Algorithmus auszuführen
	\end{itemize}
	\lstinputlisting{resources/04_wichtige_prinzipien/method_overload.cs}
\end{frame}

% ----------------------- Wert- und Referenztyp ------------------------------
\section{Wert- \& Referenztyp}
\begin{frame}{Wert- \& Referenztyp}
	\begin{itemize}
		\item C\texttt{\#} unterscheidet zwischen 2 Arten von Variablen
		\item Diese haben ein anderes Verhalten beim der Zuweisung zu einer Variable oder bei der Übergabe in eine Methode
	\end{itemize}
\end{frame}

\begin{frame}{Werttyp}
	\begin{itemize}
		\item Sind alle grundlegenden Datentypen (\alert{int}, \alert{bool}, ...) und \alert{struct}
	\end{itemize}
	\begin{itemize}
		\item Beim Kopieren oder Neuzuweisen einer Variable dieser Art wird der ganze Inhalt der Variable kopiert
		\item Es entsteht somit eine identische Kopie
		\begin{itemize}
			\item Wenn eine Kopie verändert wird, bleibt die ursprünglich variable unbetroffen
		\end{itemize}
	\end{itemize}
	\lstinputlisting{resources/04_wichtige_prinzipien/valuetype.cs}	
\end{frame}

\begin{frame}{Referenztyp}
	\begin{itemize}
		\item Sind alle Objekte von Klassen, \alert{interface} und \alert{delegate}
	\end{itemize}
	\begin{itemize}
		\item Beim Kopieren oder Neuzuweisen einer Variable dieser Art wird die Speicheradresse kopiert
		\item Da alle \glqq{}Kopien\grqq{} auf den selben Speicher zeigen, sind alle \glqq{}Kopien\grqq{} identisch
		\begin{itemize}
			\item Wenn eine \glqq{}Kopie\grqq{} verändert wird, werden alle anderen auch verändert
		\end{itemize}
	\end{itemize}
	\lstinputlisting{resources/04_wichtige_prinzipien/referencetype.cs}	
\end{frame}

% ----------------------- null ------------------------------
\section{null}
\begin{frame}{null}
	\begin{itemize}
		\item Schlüsselwort für eine Referenz die auf kein Objekt zeigt
		\item Alle Referenztyp-Variablen können \alert{null} sein
		\item Versucht man ein \alert{null} Objekt zu dereferenzieren, wird eine \alert{NullReferenceException} geworfen 
		\begin{itemize}
			\item Es ist möglich ein Objekt auf eine \alert{null} Referenz zu prüfen
		\end{itemize}
	\end{itemize}
	\lstinputlisting{resources/04_wichtige_prinzipien/null.cs}	
\end{frame}

\end{document}
