% The Slide Definitions
\input{../templates/course_definitions}

% Author and Course information
% This Document contains the information about this course.

% Authors of the slides
\author{Marc Satkowski, Sascha Peukert}

% Name of the Course
\institute{C\texttt{\#} Kurs}

% Fancy Logo
\titlegraphic{\hfill\includegraphics[height=1.25cm]{../templates/fsr_logo_cropped}}


% Presentation title
\title{Wichtige Prinzipien von C\texttt{\#}}
\date{\today}


\begin{document}

\maketitle

\begin{frame}{Gliederung}
	\setbeamertemplate{section in toc}[sections numbered]
	\tableofcontents
\end{frame}

% ----------------------- Scope ------------------------------
\section{Gültigkeit (Scope)}
\begin{frame}{Gültigkeit (Scope)}
	\begin{itemize}
		\item Variablen, Member, Klassen und andere Elementen sind nur innerhalb des Bereichs gültig in denen sie deklariert wurden:
		\begin{itemize}
			\item Methoden und Felder nur in deren Klassen
			\item Parameter nur in deren Methoden
			\item Lokale Variablen innerhalb ihrer Blöcke
		\end{itemize}
	\end{itemize}
\end{frame}

\begin{frame}{Gültigkeit (Scope)}
	\lstinputlisting{resources/04_wichtige_prinzipien/scope.cs}
\end{frame}

% ----------------------- namespace und using ------------------------------
\section{namespace \& using}
\begin{frame}{namespace \& using}
	\textbf{namespace}\\
	\begin{itemize}
		\item Werden zur Organisation von Klassen und anderem Code genutzt 
		\item Können beliebig verschachtelt werden
		\item der volle Name einer Klasse beinhaltet die Namen aller Namespaces, in denen die Klasse liegt
		\item Vermeiden von Mehrdeutigen Bezeichnen 
		\begin{itemize}
			\item Zwei Klassen dürfen den selben Namen haben, wenn diese in unterschiedlichen Namespaces liegen		
		\end{itemize}
	\end{itemize}	
	\lstinputlisting{resources/04_wichtige_prinzipien/namespace.cs}
\end{frame}

\begin{frame}{namespace \& using}
	\textbf{using}\\
	\begin{itemize}
		\item Ein Schlüsselwort zur Nutzung eines bestimmten \alert{namespace}
		\item Dies erspart die Eingabe des gesamten Namespace bei Nutzung einer Klasse
		\item Kann zu mehrdeutigen Code führen, wenn 2 Namespaces eingebunden werden, die Elemente mit selben Namen beinhalten
	\end{itemize}
	\lstinputlisting{resources/04_wichtige_prinzipien/using.cs}
\end{frame}

% ----------------------- Methodenüberladung ------------------------------
\section{Methodenüberladung}
\begin{frame}{Methodenüberladung}
	\begin{itemize}
		\item Mehrere Methoden können den selben Namen haben
		\item Diese Unterscheiden sich dann in dem Rückgabewert und/oder den Übergabeparametern
		\item Wird genutzt wenn es mehrere Wege gibt eine Berechnung/Algorithmus auszuführen
	\end{itemize}
	\lstinputlisting{resources/04_wichtige_prinzipien/method_overload.cs}
\end{frame}

% ----------------------- Wert- und Referenztyp ------------------------------
\section{Wert- \& Referenztyp}
\begin{frame}{Wert- \& Referenztyp}
	\begin{itemize}
		\item C\texttt{\#} unterscheidet zwischen 2 Arten von Variablen
		\item Diese haben ein anderes Verhalten beim der Zuweisung zu einer Variable oder bei der Übergabe in eine Methode
	\end{itemize}
\end{frame}

\begin{frame}{Werttyp}
	\begin{itemize}
		\item Sind alle grundlegenden Datentypen (\alert{int}, \alert{bool}, ...) und \alert{struct}
	\end{itemize}
	\begin{itemize}
		\item Beim Kopieren oder Neuzuweisen einer Variable dieser Art wird der ganze Inhalt der Variable kopiert
		\item Es entsteht somit eine identische Kopie
		\begin{itemize}
			\item Wenn eine Kopie verändert wird, bleibt die andere unbetroffen
		\end{itemize}
	\end{itemize}
	\lstinputlisting{resources/04_wichtige_prinzipien/valuetype.cs}	
\end{frame}

\begin{frame}{Referenztyp}
	\begin{itemize}
		\item Sind alle Objekte von Klassen, \alert{interface} und \alert{delegate}
	\end{itemize}
	\begin{itemize}
		\item Beim Kopieren oder Neuzuweisen einer Variable dieser Art wird die Speicheradresse kopiert
		\item Da alle Kopien auf den selben Speicher zeigen, sind alle Kopien identisch
		\begin{itemize}
			\item Wenn eine Kopie verändert wird, werden alle anderen auch verändert
		\end{itemize}
	\end{itemize}
	\lstinputlisting{resources/04_wichtige_prinzipien/referencetype.cs}	
\end{frame}

% ----------------------- null ------------------------------
\section{null}
\begin{frame}{null}
	\begin{itemize}
		\item Schlüsselwort für eine Referenz die auf kein Objekt zeigt
		\item Alle Referenztyp-Variablen können \alert{null} sein
		\item Versucht man ein \alert{null} Objekt zu dereferenzieren, wird eine \alert{NullReferenceException} geworfen 
		\begin{itemize}
			\item Es ist möglich ein Objekt auf eine \alert{null} Referenz zu prüfen
		\end{itemize}
	\end{itemize}
	\lstinputlisting{resources/04_wichtige_prinzipien/null.cs}	
\end{frame}

\end{document}
