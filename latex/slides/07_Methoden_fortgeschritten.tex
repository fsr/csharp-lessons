% The Slide Definitions
\input{../templates/course_definitions}

% Author and Course information
% This Document contains the information about this course.

% Authors of the slides
\author{Marc Satkowski, Sascha Peukert}

% Name of the Course
\institute{C\texttt{\#} Kurs}

% Fancy Logo
\titlegraphic{\hfill\includegraphics[height=1.25cm]{../templates/fsr_logo_cropped}}


% Presentation title
\title{Methoden (fortgeschritten)}
\date{\today}


\begin{document}

\maketitle

\begin{frame}{Gliederung}
	\setbeamertemplate{section in toc}[sections numbered]
	\tableofcontents
\end{frame}

% ----------------------- Extensions ------------------------------
\section{Extensions}
\begin{frame}{Extensions}

\end{frame}

% ----------------------- Opertatoren Überladung ------------------------------
\section{Opertatoren Überladung}
\begin{frame}{Opertatoren Überladung}

\end{frame}

% ----------------------- delegate ------------------------------
\section{delegate}
\begin{frame}{delegate}

\end{frame}

% ----------------------- Lamda ------------------------------
\section{Lamda}
\begin{frame}{Lamda}

\end{frame}

% ----------------------- Schlüsselwörter ------------------------------
\section{Schlüsselwörter}
\begin{frame}{Schlüsselwörter}
	\textbf{ref}\\
	\begin{itemize}
		\item Ermöglicht es einen Werttyp (wie z.B. int) als Referenztyp in eine Methode zu übergeben
		\item Wird dabei vor den Datentyp eines Übergabeparameters
		\item Und beim Aufruf vor die Variable geschrieben
	\end{itemize}
	\textbf{params}\\
	\begin{itemize}
		\item Ermöglicht es eine beliebige Anzahl von Übergabeparametern eines Typs einer Methode zu geben
		\item Diese werden in der Methode dann als Array genutzt
		\item Die Übergabe kann über eine mit Komma getrennte Liste oder über ein Array passieren
	\end{itemize}
\end{frame}

\begin{frame}{Schlüsselwörter}
	\lstinputlisting{resources/07_Methoden_fortgeschritten/ref_param_1.cs}
	\lstinputlisting{resources/07_Methoden_fortgeschritten/ref_param_2.cs}
\end{frame}

\begin{frame}{Schlüsselwörter}
	\textbf{out}\\
	\begin{itemize}
		\item Eine Variable wird, wie bei \alert{ref}, als Verweis übergeben
		\item Jedoch muss diese Variable nun nicht deklariert sein
		\item Die Methode wird dann gezwungen während ihrer Ausführung der \alert{out}-Variable etwas zuzuweisen
	\end{itemize}
\end{frame}

\begin{frame}{Schlüsselwörter}
	\lstinputlisting{resources/07_Methoden_fortgeschritten/out_1.cs}
	\lstinputlisting{resources/07_Methoden_fortgeschritten/out_2.cs}
\end{frame}

\end{document}
