% The Slide Definitions
\input{../templates/course_definitions}

% Author and Course information
% This Document contains the information about this course.

% Authors of the slides
\author{Marc Satkowski, Sascha Peukert}

% Name of the Course
\institute{C\texttt{\#} Kurs}

% Fancy Logo
\titlegraphic{\hfill\includegraphics[height=1.25cm]{../templates/fsr_logo_cropped}}


% Presentation title
\title{Objektorientierung}
\date{\today}


\begin{document}

\maketitle

\begin{frame}{Gliederung}
	\setbeamertemplate{section in toc}[sections numbered]
	\tableofcontents
\end{frame}

% ----------------------- Objektorientierung ------------------------------
\section{Objektorientierung}
\begin{frame}{Objektorientierung}
	\begin{itemize}
		\item Alles in C\texttt{\#} ist ein Objekt, auch die primitiven Datentypen, wie z.B. int oder bool
		\item Damit kann alles in C\texttt{\#} Attribute und Methoden besitzen		
	\end{itemize}
\end{frame}

\subsection{Klassen}
\begin{frame}{Klassen}
	\begin{itemize}
		\item Ist eine Vorlage oder ein Bauplan
		\item Kann Felder, Eigenschaften und Methoden besitzen
		\item Klassen werden durch das Schlüsselwort \alert{class} gekennzeichnet
	\end{itemize}	
	\lstinputlisting{resources/03_objektorientierung/class.cs}
\end{frame}

\subsection{Objekte}
\begin{frame}{Objekte}
	\begin{itemize}
		\item Werden nach Klassen erstellt und sind somit Instanzen einer Klasse
		\begin{itemize}
			\item Dies geschieht mit dem Schlüsselwort \alert{new} und einem von der Klasse definierten Konstruktor
		\end{itemize}
		\item Haben in den von der Klasse vorgeschriebenen Attributen konkrete Werte
	\end{itemize}
	\lstinputlisting{resources/03_objektorientierung/objects.cs}	
\end{frame}

\subsection{Konstruktor}
\begin{frame}{Konstruktor}
	\begin{itemize}
		\item Sind spezielle Methoden die beim Erstellen einer Instanz einer Klasse aufgerufen werden müssen
		\item Stellen Basis für das Existieren und Funktionieren des Objektes her 
		\item Sie besitzen keinen Namen und haben als Rückgabetyp ein Objekt der eigenen Klasse
		\item Falls kein eigener Konstruktor erstellt wurde, existiert immer der leere Konstruktor (der ohne Übergabeparameter)
	\end{itemize}
	\lstinputlisting{resources/03_objektorientierung/constructor.cs}	
\end{frame}

\begin{frame}{Konstruktor}
	\textbf{Schlüsselwort new}\\
	\begin{itemize}
		\item Zur Erstellung einer Instanz einer Klasse
		\item Wird immer mit einem Konstruktor zusammen genutzt
	\end{itemize}
	\lstinputlisting{resources/03_objektorientierung/new.cs}
\end{frame}

\subsection{Felder \& Eigenschaften}
\begin{frame}{Felder \& Eigenschaften}
	\textbf{Felder}\\
	\begin{itemize}
		\item Werden auch Attribute genannt
		\item Sind Variablen eines Typs, welche direkt in einer Klasse deklariert wurden
	\end{itemize}
	\textbf{Eigenschaften}\\
	\begin{itemize}
		\item Bietet flexible Mechanismen zum Lesen, Schreiben und Berechnen der Werte eines Feldes
		\item Besitzen nach dem Namen einen Block in denen ein \alert{get} und \alert{set} definiert werden kann
		\begin{itemize}
			\item Diese agieren wie Methoden und haben somit einen Körper
			\item \alert{set} hat dabei einen Übergabeparameter, und zwar \alert{value}
			\begin{itemize}
				\item Dieser ist vom selben Typ, wie die Eigenschaft
			\end{itemize}
		\end{itemize}
	\end{itemize}
\end{frame}

\begin{frame}{Felder \& Eigenschaften}	
	\lstinputlisting{resources/03_objektorientierung/field_property.cs}
\end{frame}

% ----------------------- Zugriffsmodifikatoren/Sichtbarkeiten ------------------------------
\section{Zugriffsmodifikatoren/Sichtbarkeiten}
\begin{frame}{Zugriffsmodifikatoren/Sichtbarkeiten}
	\begin{itemize}
		\item Geben an, ob der zugehörige Code von anderen Code-Abschnitten verwendet werden kann
		\item Es gibt mehrere Arten
		\begin{description}
			\item[public] kann von überall zugriffen werden, solange darauf verweist wird
			\item[private] kann nur von Code in der eigenen Klasse benutzt werden
			\item[protected] kann nur von Code in der eigenen und von abgeleiteten Klassen benutzt werden
			\item[internal]
		\end{description}
	\end{itemize}
\end{frame}

\begin{frame}{Zugriffsmodifikatoren/Sichtbarkeiten}
	\lstinputlisting{resources/03_objektorientierung/access_modifier.cs}
\end{frame}

% ----------------------- namespace und using ------------------------------
\section{namespace \& using}
\begin{frame}{namespace \& using}
	\textbf{namespace}\\
	\begin{itemize}
		\item Werden zur Organisation von Klassen und anderem Code genutzt
	\end{itemize}	
	\lstinputlisting{resources/03_objektorientierung/namespace.cs}
\end{frame}

\begin{frame}{namespace \& using}
	\textbf{using}\\
	\begin{itemize}
		\item Ein Schlüsselwort zur Nutzung eines bestimmten namespace
		\item Dies erspart die Eingabe des gesamten namespace bei Nutzung einer Klasse
	\end{itemize}
	\lstinputlisting{resources/03_objektorientierung/using.cs}
\end{frame}

% ----------------------- Dokumentationskommentare ------------------------------
\section{Dokumentationskommentare}
\begin{frame}{Dokumentationskommentare}
	\begin{itemize}
		\item Das Kommentar beginnt mit 3, anstatt 2 \alert{/}
		\item Beschreiben nur Klassen, Felder/Eigenschaften und Methoden
		\begin{itemize}
			\item Das Kommentar steht dabei immer direkt über den zu kommentierenden Code
		\end{itemize}
		\item Man kann aus diesen eine Dokumentation generieren
		\item Außerdem können einige IDEs (z.B. Visual Studio) daraus eine Kurzbeschreibung der Elemente während des Programmierens schaffen		
	\end{itemize}
\end{frame}

\begin{frame}{Dokumentationskommentare}
	\textbf{Tags}
	\begin{itemize}
		\item Geben an, was genau von dem Code dokumentiert werden soll
		\item 3 wichtige Tags:
		\begin{description}
			\item[summary] Beschreibung des zu kommentierenden Codes\\ Für alle 3 Arten
			\item[param] Beschreibung der Übergabeparameter einer Methode
			\item[returns] Beschreibung des Rückgabewertes einer Methode
		\end{description}
		\item \href{https://msdn.microsoft.com/de-de/library/5ast78ax.aspx}{Hier findet man weitere Tags}
	\end{itemize}
\end{frame}

\begin{frame}{Dokumentationskommentare}
	\begin{itemize}
		\item Die genannten Tags am Beispiel einer Methode:
	\end{itemize}
	\lstinputlisting{resources/03_objektorientierung/documentation.cs}	
\end{frame}

\end{document}
