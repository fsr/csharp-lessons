% The Slide Definitions
\input{../templates/course_definitions}

% Author and Course information
% This Document contains the information about this course.

% Authors of the slides
\author{Marc Satkowski, Sascha Peukert}

% Name of the Course
\institute{C\texttt{\#} Kurs}

% Fancy Logo
\titlegraphic{\hfill\includegraphics[height=1.25cm]{../templates/fsr_logo_cropped}}


% Presentation title
\title{Objektorientierung}
\date{\today}


\begin{document}

\maketitle

\begin{frame}{Gliederung}
	\setbeamertemplate{section in toc}[sections numbered]
	\tableofcontents
\end{frame}

% ----------------------- Verzweigung ------------------------------
\section{Verzweigungen}
\subsection{Tertiäre-Verzweigung}
\begin{frame}{Tertiäre-Verzweigung}
	\begin{itemize}
		\item Werden genauso genutzt wie eine if-else-Verzweigung
		\item Syntax:
		\begin{itemize}
			\item \texttt{\alert{<Bedingung>} ? \alert{<Dann-Pfad>} : \alert{<Sonst-Pfad>};}
		\end{itemize}
		\item Das \alert{\texttt{?}} steht für \alert{dann}
		\item Das \alert{\texttt{:}} steht für \alert{sonst}
	\end{itemize}		
	\lstinputlisting{resources/03_objektorientierung/tertiary.cs}
\end{frame}

\subsection{switch case}
\begin{frame}{switch case}
	\begin{itemize}
		\item Hat ähnliches Verhalten wie die bedingte Verzweigung
		\begin{itemize}
			\item \alert{default} verhält sich wie \alert{else}
		\end{itemize}
		\item Jeder \alert{case} prüft die Variable im \alert{switch} auf Gleichheit und führt den darunterliegenden Code bis zum \alert{break} aus
	\end{itemize}
	\lstinputlisting{resources/03_objektorientierung/switch_case.cs}
\end{frame}

% ----------------------- Schleifen ------------------------------
\section{Schleifen}
\subsection{Zählschleife (for)}
\begin{frame}{Zählschleife (for)}
	\begin{itemize}
		\item Zählt über eine Zählvariable in gewisser Schrittweite, bis die Bedingung nicht mehr erfüllt ist
		\item Syntax:
		\begin{itemize}
			\item \texttt{for( \alert{<Zählvariable>}; \alert{<Bedingung>}; \alert{<Schrittweite>} )\\ \{ \alert{<Code>} \}}
		\end{itemize}
	\end{itemize}
	\lstinputlisting{resources/03_objektorientierung/for.cs}	
\end{frame}

\subsection{break \& continue}
\begin{frame}{break \& continue}
	\textbf{break}\\
	\begin{itemize}
		\item Beendet die direkt umschließende Schleife, in der die Anweisung auftritt
		\item Beachtet dabei nicht wie viele Schleifendurchläufe es eigentlich noch geben würde
	\end{itemize}
	\textbf{continue}\\
	\begin{itemize}
		\item Beendet den jetzigen Schleifen-Block-Durchlauf und geht zur Bedingung der Schleife zurück
		\item Daraufhin kann die Schleife mit dem nächsten Iterationsschritt fortsetzen
	\end{itemize}
\end{frame}

\begin{frame}{break \& continue}
	\textbf{break \& continue:}\\
	\begin{itemize}
		\item Beide können und sollten mit Verzweigungen benutzt werden
	\end{itemize}
	\lstinputlisting{resources/03_objektorientierung/break_continue.cs}
\end{frame}

% ----------------------- Objektorientierung ------------------------------
\section{Objektorientierung}
\begin{frame}{Objektorientierung}
	\begin{itemize}
		\item Alles in C\texttt{\#} ist ein Objekt, auch die primitiven Datentypen, wie z.B. \alert{int} oder \alert{bool}
		\item Damit kann alles in C\texttt{\#} Felder und Methoden besitzen		
	\end{itemize}
\end{frame}

\subsection{Klassen}
\begin{frame}{Klassen}
	\begin{itemize}
		\item Ist eine Vorlage oder ein Bauplan
		\item Kann Felder, Eigenschaften und Methoden besitzen
		\item Klassen werden durch das Schlüsselwort \alert{class} gekennzeichnet
	\end{itemize}	
	\lstinputlisting{resources/03_objektorientierung/class.cs}
\end{frame}

\subsection{Objekte}
\begin{frame}{Objekte}
	\begin{itemize}
		\item Werden anhand von Klassen erstellt und sind somit Instanzen einer Klasse
		\begin{itemize}
			\item Dies geschieht mit dem Schlüsselwort \alert{new} und einem von der Klasse definierten Konstruktor
		\end{itemize}
		\item Sollten die Felder einer Klasse mit konkreten Werten befüllen
	\end{itemize}
	\lstinputlisting{resources/03_objektorientierung/objects.cs}	
\end{frame}

\subsection{Konstruktor}
\begin{frame}{Konstruktor}
	\begin{itemize}
		\item Sind spezielle Methoden die beim Erstellen einer Instanz einer Klasse aufgerufen werden
		\item Sie besitzen keinen Namen und haben als Rückgabetyp ein Objekt der eigenen Klasse
		\begin{itemize}
			\item Es wird aber \textbf{kein} \alert{return} benötigt
		\end{itemize}
		\item Sollte sicherstellen, dass eine gültige Instanz einer Klasse erzeugt wird
		\item Falls kein eigener Konstruktor deklariert wurde, existiert immer der Standardkonstruktor (parameterlos und weist allen Feldern Standardwerte zu)
	\end{itemize}
	\lstinputlisting{resources/03_objektorientierung/constructor.cs}	
\end{frame}

\begin{frame}{Konstruktor}
	\textbf{Schlüsselwort new}\\
	\begin{itemize}
		\item Zur Erstellung einer Instanz einer Klasse
		\item Ruft immer einen Konstruktor der jeweiligen Klasse auf
	\end{itemize}
	\lstinputlisting{resources/03_objektorientierung/new.cs}
\end{frame}

\subsection{Felder \& Eigenschaften}
\begin{frame}{Felder \& Eigenschaften}
	\textbf{Felder}\\
	\begin{itemize}
		\item Werden auch Attribute genannt
		\item Sind Variablen eines beliebigen Typs, welche direkt in einer Klasse deklariert wurden und damit dieser Klasse zugehörig sind
	\end{itemize}
	\textbf{Eigenschaften}\\
	\begin{itemize}
		\item Bietet flexible Mechanismen zum Lesen, Schreiben und Berechnen der Werte eines Feldes
		\item Es muss aber nicht zwangsläufig ein Feld abbilden
		\item Besitzen nach dem Namen einen Block in denen ein \alert{get} und/oder \alert{set} definiert werden kann
	\end{itemize}
\end{frame}

\begin{frame}{Felder \& Eigenschaften}
	\textbf{Eigenschaften: get \& set}
	\begin{itemize}
		\item Diese agieren wie Methoden
		\item Sie können einen Körper haben, müssen es aber nicht
		\begin{itemize}
			\item Sobald einer der beiden einen Körper hat, muss der andere auch einen haben (wenn sowohl \alert{get} als auch \alert{set} implementiert werden sollen)
			\item Falls kein eigener Körper definiert wurde, wird die Variable wie ein Feld benutzt
		\end{itemize}
		\item Besitzen eigene Sichtbarkeiten (unterschiedlich vom Feld selbst)
		\item \alert{set} hat dabei einen Übergabeparameter, und zwar \alert{value}
		\begin{itemize}
			\item Dieser ist vom selben Typ, wie die Eigenschaft		
		\end{itemize}		
	\end{itemize}
\end{frame}

\begin{frame}{Felder \& Eigenschaften}	
	\lstinputlisting{resources/03_objektorientierung/field_property.cs}
\end{frame}

\subsection{Zugriff}
\begin{frame}{Zugriff}
	\textbf{Zugriff mit \texttt{\alert{.}} (Punkt)}\\
	\begin{itemize}
		\item Um auf Inhalte (Methoden, Eigenschaften, ...) eines Objektes zugreifen zu können, nutzt man einen Punkt \alert{\textbf{.}}
		\item Dahinter kommt der Member den man nutzen möchte
	\end{itemize}
	\textbf{this}\\
	\begin{itemize}
		\item Gibt die aktuelle Instanz der Klasse (in dem man gerade ist und arbeitet) als Variable wieder
		\item Nutzen:
		\begin{itemize}
			\item Übergabe der aktuellen Instanz der Klasse in Methoden
			\item Nutzung von Membern, die durch gleiche Kennzeichner ausgeblendet werden (falls Member und Variablen den gleichen Namen haben)
			\item Und andere (wird später erläutert)
		\end{itemize}
	\end{itemize}
\end{frame}

% ----------------------- Scope ------------------------------
\section{Gültigkeit (Scope)}
\begin{frame}{Gültigkeit (Scope)}
	\begin{itemize}
		\item Variablen, Member, Klassen und andere Elementen sind nur innerhalb des Bereichs gültig in denen sie deklariert wurden:
		\begin{itemize}
			\item Methoden und Felder nur in deren Klassen
			\item Parameter nur in deren Methoden
			\item Lokale Variablen innerhalb ihrer Blöcke
		\end{itemize}
	\end{itemize}
\end{frame}

\begin{frame}{Gültigkeit (Scope)}
	\lstinputlisting{resources/03_objektorientierung/scope.cs}
\end{frame}

\end{document}
