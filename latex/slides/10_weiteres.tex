% The Slide Definitions
\input{../templates/course_definitions}

% Author and Course information
% This Document contains the information about this course.

% Authors of the slides
\author{Marc Satkowski, Sascha Peukert}

% Name of the Course
\institute{C\texttt{\#} Kurs}

% Fancy Logo
\titlegraphic{\hfill\includegraphics[height=1.25cm]{../templates/fsr_logo_cropped}}


% Presentation title
\title{Weiteres ins C\texttt{\#}}
\date{\today}


\begin{document}

\maketitle

\begin{frame}{Gliederung}
	\setbeamertemplate{section in toc}[sections numbered]
	\tableofcontents
\end{frame}

% ----------------------- Eventhandling ------------------------------
\section{Event Handling}
\subsection{Event Handler}
\begin{frame}{EventHandler}
	\begin{itemize}
		\item Ein vordefinierter Delegate
		\item Wird genutzt um Eventhandling zu betreiben
		\begin{itemize}
			\item Man kann auch einen eigenen nehmen
		\end{itemize}
		\item Gibt \alert{void} zurück
		\item Hat 2 Übergabeparameter:
		\begin{itemize}
			\item \alert{object} (der Sender dieses Events)
			\item \alert{TEventArgs}
		\end{itemize}
	\end{itemize}
	\begin{itemize}
		\item Der Kopf des EventHandlers im Framework
	\end{itemize}
	\lstinputlisting{resources/10_weiteres/event_handler.cs}	
\end{frame}

\begin{frame}{EventArgs}
	\begin{itemize}
		\item Gibt dem Event die Möglichkeit zusätzliche Informationen an Empfänger zu übergeben
		\begin{itemize}
			\item Falls keine Informationen zum Event zur Verfügung stehen kann man \alert{EventArgs.Empty} übergeben
		\end{itemize}
		\item Da der EventHanlder generisch ist, kann man seine eigenen \alert{EventArgs} übergeben
	\end{itemize}
	\lstinputlisting{resources/10_weiteres/event_args.cs}
\end{frame}

\subsection{event}
\begin{frame}{event}
	\begin{itemize}
		\item Wird genutzt um bei einem Publisher ein Event zu deklarieren
		\item Zusammen mit Attributen von einem Delegatetype genutzt
		\item Kann ebenfalls alle Modifikatoren (wie \alert{virtual} oder \alert{override} nutzen)
		\begin{itemize}
			\item Bei \alert{abstract} muss ein eigener add/remove-Ereignisaccessorblock erstellt werden
		\end{itemize}
	\end{itemize}				
	\lstinputlisting{resources/10_weiteres/event.cs}
\end{frame}

\subsection{Mehrere Events}
\begin{frame}{Mehrere Events}
	\begin{itemize}
		\item Man kann verschiedene Methoden einen EventHandler (oder anderen \alert{Delegate}) mit dem add-Ereignisaccessorblock
			\begin{itemize}
				\item \texttt{\alert{+=}}
			\end{itemize}
		\item Entfernt können diese wieder mit dem remove-Ereignisaccessorblock
			\begin{itemize}
				\item \texttt{\alert{-=}}
			\end{itemize}		
		\item Die Methoden werden dann nach einander ausgeführt
	\end{itemize}
\end{frame}

\subsection{Beispiel}
\begin{frame}{Beispiel}
	\lstinputlisting{resources/10_weiteres/event_handling_creation.cs}
\end{frame}

\begin{frame}{Beispiel}
	\lstinputlisting{resources/10_weiteres/event_handling_usage_1.cs}
\end{frame}

\begin{frame}{Beispiel}
	\lstinputlisting{resources/10_weiteres/event_handling_usage_2.cs}
\end{frame}

% ----------------------- Weiteres zu null ------------------------------
\section{Weiteres zu null}
\subsection{Nullable}
\begin{frame}{Nullable}
	\begin{itemize}
		\item Werden genutzt, damit Werttyp-Variablen ebenfalls den Inhalt von \alert{null} haben können
		\item Ist eine generische Klasse
		\item Besitzt 2 Eigenschaften
		\begin{description}
			\item[HasValue] Gibt an, ob das Objekt einen Wert hat oder nicht
			\item[Value] Gibt den Wert des Objektes zurück\\ Wirft eine InvalidOperationException falls HasValue \alert{false} ist 
		\end{description}
		\item Syntax:
		\begin{itemize}
			\item \texttt{\alert{<Werttyp>}? \alert{<Variablenname>};} oder
			\item \texttt{Nullable<\alert{<Datentyp>}> \alert{<Variablenname>};}
		\end{itemize}
	\end{itemize}
\end{frame}

\begin{frame}{Nullable}
	\lstinputlisting{resources/10_weiteres/nullable.cs}	
\end{frame}

\begin{frame}{Weiteres zu null}
	\begin{itemize}
		\item Falls ein Objekt \alert{null} sein kann sollte man diese vor dem benutzen immer überprüfen
		\item Da \alert{obj != null} viel Schreibaufwand ist, kann dies vereinfacht werden	
	\end{itemize}
	\begin{itemize}
		\item Man nutzt dafür \texttt{\alert{?.}}
		\item Syntax:
		\begin{itemize}
			\item \texttt{obj\alert{?.}Property ...}
		\end{itemize}
		\item Es wird vor dem Benutzen des Objektes geschaut, ob dieses \alert{null} ist
		\item Falls es nich \alert{null} ist, wird es benutzt, sonst gibts es \alert{null} zurück
	\end{itemize}
\end{frame}

\begin{frame}{Weiteres zu null}
	\lstinputlisting{resources/10_weiteres/null1.cs}
	\begin{itemize}
		\item Hier für einen \alert{delegate}
	\end{itemize}		
	\lstinputlisting{resources/10_weiteres/null2.cs}	
\end{frame}

% ----------------------- LINQ ------------------------------
\section{LINQ}
\begin{frame}{LINQ}
	\begin{itemize}
		\item Language-Integrated Query
		\item LINQ ermöglicht es Abfragen an Objekte und Datenstrukturen zu stellen
		\item Ähnlich zu z.B. SQL
	\end{itemize}
\end{frame}

\begin{frame}{LINQ}
	\lstinputlisting{resources/10_weiteres/linq.cs}	
\end{frame}

\end{document}
