% The Slide Definitions
\input{../templates/course_definitions}

% Author and Course information
% This Document contains the information about this course.

% Authors of the slides
\author{Marc Satkowski, Sascha Peukert}

% Name of the Course
\institute{C\texttt{\#} Kurs}

% Fancy Logo
\titlegraphic{\hfill\includegraphics[height=1.25cm]{../templates/fsr_logo_cropped}}


% Presentation title
\title{Methoden (fortgeschritten)}
\date{\today}


\begin{document}

\maketitle

\begin{frame}{Gliederung}
	\setbeamertemplate{section in toc}[sections numbered]
	\tableofcontents
\end{frame}

% ----------------------- Extensions ------------------------------
\section{Extensions}
\begin{frame}{Extensions}
	\begin{itemize}
		\item Ermöglichen es vorhandenen Typen eigene Methoden hinzuzufügen
		\item Ohne eine abgeleitete Klasse zu erstellen oder den originalen Typen neu zu erstellen
		\item Sind statische Methoden, welche aber als Instanzmethoden auf den erweiterten Typ agieren
		\item Befinden sich selbst in einer extra statischen Klasse
		\item Syntax:
		\begin{itemize}
			\item \texttt{public static \alert{<Rückgabetyp> <Name>}( this \alert{<Erweiterter Typ> <Variablenname>}, \alert{[Übergabeparameter]} )\\ \{ \alert{<Code>} \}}
		\end{itemize}
	\end{itemize}
\end{frame}

\begin{frame}{Extensions}
	\lstinputlisting{resources/09_methoden_fortgeschritten_2/extensions.cs}
\end{frame}

% ----------------------- yield ------------------------------
\section{yield}
\begin{frame}{yield}
	\begin{itemize}
		\item Schlüsselwort
		\item Macht aus einer Methode, Operator oder get-Accessor einen Iterator
		\begin{itemize}
			\item Der Rückgabetyp muss \alert{IEnumerable}, \alert{IEnumerable<T>}, \alert{IEnumerator} oder \alert{IEnumerator<T>} sein
		\end{itemize}
		\item Syntax:
		\begin{itemize}
			\item \texttt{\alert{yield} return \alert{<Rückgabewert>};}
			\item \texttt{\alert{yield} break;}
		\end{itemize}
	\end{itemize}
\end{frame}

\begin{frame}{yield}
	\lstinputlisting{resources/09_methoden_fortgeschritten_2/yield.cs}	
\end{frame}

% ----------------------- Schlüsselwörter ------------------------------
\section{Schlüsselwörter}
\begin{frame}{Schlüsselwörter}
	\textbf{ref}\\
	\begin{itemize}
		\item Ermöglicht es einen Werttyp (wie z.B. int) als Referenztyp in eine Methode zu übergeben
		\item Wird dabei vor den Datentyp eines Übergabeparameters
		\item Und beim Aufruf vor die Variable geschrieben
	\end{itemize}
	\textbf{params}\\
	\begin{itemize}
		\item Ermöglicht es eine beliebige Anzahl von Übergabeparametern eines Typs einer Methode zu geben
		\item Diese werden in der Methode dann als Array genutzt
		\item Die Übergabe kann über eine mit Komma getrennte Liste oder über ein Array passieren
	\end{itemize}
\end{frame}

\begin{frame}{Schlüsselwörter}
	\lstinputlisting{resources/09_methoden_fortgeschritten_2/ref_param_creation.cs}
	\lstinputlisting{resources/09_methoden_fortgeschritten_2/ref_param_using.cs}
\end{frame}

\begin{frame}{Schlüsselwörter}
	\textbf{out}\\
	\begin{itemize}
		\item Eine Variable wird, wie bei \alert{ref}, als Verweis übergeben
		\item Jedoch muss diese Variable nun nicht deklariert sein
		\item Die Variable enthält zum Ende der Methode den zuletzt zugewiesenen Wert, es muss innerhalb der Methode mindestens eine Zuweisung erfolgen
		\item kann verwendet werden, wenn eine Methode mehr als einen Wert zurückgeben soll
	\end{itemize}
\end{frame}

\begin{frame}{Schlüsselwörter}
	\lstinputlisting{resources/09_methoden_fortgeschritten_2/out_creation.cs}
	\lstinputlisting{resources/09_methoden_fortgeschritten_2/out_using.cs}
\end{frame}

\end{document}
