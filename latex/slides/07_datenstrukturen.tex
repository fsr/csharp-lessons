% The Slide Definitions
\input{../templates/course_definitions}

% Author and Course information
% This Document contains the information about this course.

% Authors of the slides
\author{Marc Satkowski, Sascha Peukert}

% Name of the Course
\institute{C\texttt{\#} Kurs}

% Fancy Logo
\titlegraphic{\hfill\includegraphics[height=1.25cm]{../templates/fsr_logo_cropped}}


% Presentation title
\title{Datenstrukturen}
\date{\today}


\begin{document}

\maketitle

\begin{frame}{Gliederung}
	\setbeamertemplate{section in toc}[sections numbered]
	\tableofcontents
\end{frame}

% ----------------------- Indexer ------------------------------
\section{Indexer}
\begin{frame}{Indexer}
	\begin{itemize}
		\item Sind Member einer Klasse
		\item Geben die Möglichkeit wie ein/e Array/Collection über \alert{[ ]} angesprochen zu werden
		\item Brauchen immer ein \alert{set} und \alert{get}
		\item Syntax:
		\begin{itemize}
			\item \texttt{\alert{<Typ>} this[\alert{<Indextyp> <Indexname>}]\\ \{ get \{ \alert{<Code>} \} set \{ \alert{<Code>}\} \}}
		\end{itemize}
	\end{itemize}
\end{frame}

\begin{frame}{Indexer}
	\lstinputlisting{resources/07_datenstrukturen/indexer.cs}
\end{frame}

% ----------------------- Enummeration ------------------------------
\section{Enumeration}
\begin{frame}{Enumeration}
	\begin{itemize}
		\item Zu dt. Aufzählung
		\item Das Schlüsselwort dafür ist \alert{enum}
		\item Ist ein eigens erstellter Datentyp mit fester Anzahl von Konstanten
		\item Der Konstante werden Intern ein \alert{int}-Wert zugeordnet
		\begin{itemize}
			\item Man kann auch andere Typen angeben (wie z.B. \alert{byte}, \alert{short} ...)
			\item Diese beginnt Standardmäßig bei 0 und wird mit jedem neuen Element um 1 erhöht
			\item Man kann den Konstanten Werte zuweisen und Werte können mehrmals vergeben werden
			\item Man kann damit einen \alert{enum}-Wert in einen zugehörigen Typ Wert casten			
		\end{itemize}
		\item Syntax zur Nutzung:
		\begin{itemize}
			\item \texttt{\alert{<Enum> <Variablenname>} = \alert{<Enum>}.\alert{<Enumwert>};} oder
			\item \texttt{\alert{<Enum> <Variablenname>} = \alert{<Int-Wert>};}
		\end{itemize}
	\end{itemize}
\end{frame}

\begin{frame}{Enumeration}
	\lstinputlisting{resources/07_datenstrukturen/enum_creation.cs}
	\lstinputlisting{resources/07_datenstrukturen/enum_using.cs}
\end{frame}

% ----------------------- Structs ------------------------------
\section{Structs}
\begin{frame}{Structs}
	\begin{itemize}
		\item Schlüsselwort ist \alert{struct}
		\item Ist nahezu identisch mit einer Klasse
		\item Sie sind nicht Referenz-, sondern Werttypen
		\item Werden genutzt um kleine Gruppen verwandter Variablen zusammen zufassen
		\item Sie können von Interfaces, aber nicht von anderen Structs oder Klassen erben 
		\begin{itemize}
			\item \alert{struct} sind automatisch \alert{sealed}
		\end{itemize}
	\end{itemize}
\end{frame}

\begin{frame}{Structs}
	\lstinputlisting{resources/07_datenstrukturen/struct_creation.cs}
	\lstinputlisting{resources/07_datenstrukturen/struct_using.cs}
\end{frame}

% ----------------------- Schlüsselwörter ------------------------------
\section{Schlüsselwörter}
\begin{frame}{Schlüsselwörter}
	\textbf{static}\\
	\begin{itemize}
		\item Es können Klassen, Methoden, Eigenschaften, Felder und mehr statisch sein
		\item Von einer statischen Klasse kann keine Instanz erzeugt werden
		\begin{itemize}
			\item Diese wird automatisch bei Programmstart angelegt
		\end{itemize}
		\item Statische Member existieren außerhalb eines Objektes einer Klasse
		\begin{itemize}
			\item Werden über die Klasse/Datentyp aufgerufen
			\item Können nicht auf nicht-statische Member der zugehörigen Klasse zugreifen
		\end{itemize}
	\end{itemize}
	\textbf{readonly}\\
	\begin{itemize}
		\item Kann nur auf Felder benutzt werden
		\item Stellt sicher, dass das Feld im Laufe der Benutzung nicht verändert wird
		\item Der Wert des Feldes kann nur durch direkte Deklaration oder in einem Konstruktor der zugehörigen Klasse initialisiert werden
	\end{itemize}
\end{frame}

\begin{frame}{Schlüsselwörter}
	\lstinputlisting{resources/07_datenstrukturen/static_readonly.cs}
\end{frame}

\begin{frame}{Schlüsselwörter}
	\textbf{var}\\
	\begin{itemize}
		\item Kann für Variablen innerhalb von Methoden genutzt werden
		\item Dabei wird der Datentyp der Variable durch den Compiler bestimmt
		\begin{itemize}
			\item Der Datentyp muss zur Compilezeit bekannt sein
		\end{itemize}
		\item Syntax:
		\begin{itemize}
			\item \texttt{var \alert{<Variablenname>} = \alert{<Eine Zuweisung>};}
		\end{itemize}
		\item Man sollte es nur nutzen, wenn man durch die rechte Seite erkennt, was für ein Typ zugewiesen wird
	\end{itemize}
	\lstinputlisting{resources/07_datenstrukturen/var.cs}
\end{frame}

\end{document}
