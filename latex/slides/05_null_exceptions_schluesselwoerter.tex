% The Slide Definitions
\input{../templates/course_definitions}

% Author and Course information
% This Document contains the information about this course.

% Authors of the slides
\author{Marc Satkowski, Sascha Peukert}

% Name of the Course
\institute{C\texttt{\#} Kurs}

% Fancy Logo
\titlegraphic{\hfill\includegraphics[height=1.25cm]{../templates/fsr_logo_cropped}}


% Presentation title
% TODO Change the topic of the lesson
\title{Awesome Stuff}
\date{\today}


\begin{document}

\maketitle

\begin{frame}{Gliederung}
	\setbeamertemplate{section in toc}[sections numbered]
	\tableofcontents
\end{frame}

% ----------------------- null ------------------------------
\section{null}
\begin{frame}{null}
	\begin{itemize}
		\item Schlüsselwort für eine Referenz die auf kein Objekt zeigt
		\item Alle Referenztyp-Variablen können \alert{null} sein
		\item Mit/auf \alert{null} kann man nicht arbeiten, sonst gibt es eine ArgumentNullException
		\begin{itemize}
			\item Außer ein Vergleich ob eine Variable \alert{null} ist
		\end{itemize}
	\end{itemize}
	\lstinputlisting{resources/05_null_exceptions_schluesselwoerter/null.cs}	
\end{frame}

\subsection{Nullable}
\begin{frame}{Nullable}
	\begin{itemize}
		\item Werden genutzt, damit Werttyp-Variablen ebenfalls \alert{null} haben können
		\item Besitzt 2 Eigenschaften
		\begin{description}
			\item[HasValue] Gibt an, ob das Objekt einen Wert hat oder nicht
			\item[Value] Gibt den Wert des Objektes zurück\\ Wirft eine InvalidOperationException falls HasValue false ist 
		\end{description}
		\item Syntax:
		\begin{itemize}
			\item \texttt{\alert{<Werttyp>}? \alert{<Variablenname>};} oder
			\item \texttt{Nullable<\alert{<Datentyp>}> \alert{<Variablenname>};}
			\begin{itemize}
				\item Das ist ein Generic
			\end{itemize}
		\end{itemize}
	\end{itemize}
\end{frame}

\begin{frame}{Nullable}
	\lstinputlisting{resources/05_null_exceptions_schluesselwoerter/nullable.cs}	
\end{frame}

% ----------------------- Exceptions ------------------------------
\section{Exceptions}
\begin{frame}{Exceptions}
	\begin{itemize}
		\item Zu dt. Ausnahmen
		\item Zeigen an wenn im Programm eine Situation aufgetreten ist, die es nicht lösen kann
		\item Beinhalten Information über Art und Auftreten des Fehlers
		\item Es gibt verschiedene Arten, wie z.B.:
		\begin{itemize} 
			\item ArgumentException
			\item ArgumentNullException	
			\item FileNotFoundException
		\end{itemize}
		\item Sie bringen das Programm zum Abstürzen, wenn diese nicht vom Code behandelt werden
	\end{itemize}
\end{frame}

\subsection{Handling}
\begin{frame}{Handling}
	\textbf{try catch finally}\\
	\begin{itemize}
		\item Wird genutzt um Exceptions aufzufangen und diese zu behandeln
		\item Der im \alert{try}-Block stehende Code wird ausgeführt und bei auftreten einer Exception werden die \alert{catch}-Blöcke ausgeführt
		\item Es kann mehrere \alert{catch}-Blöcke für verschiedene Exception geben
		\begin{itemize}
			\item Die Exception kann im \alert{catch}-Block als Variable genutzt werde
			\item Man geht immer vom spezifischen zum allgemeinem Fehler, damit alle Exceptions gefangen werden können
		\end{itemize}
		\item Der \alert{finally}-Block wird immer ausgeführt, auch wenn keine Exception aufgetreten ist
		\begin{itemize}
			\item Wird auch Bereinigungsblock genannt
		\end{itemize}
	\end{itemize}
\end{frame}

\begin{frame}{Handling}
	\lstinputlisting{resources/05_null_exceptions_schluesselwoerter/exception_handling.cs}	
\end{frame}

\subsection{Eigene Exceptions}
\begin{frame}{Eigene Exceptions}
	\textbf{throw}\\
	\begin{itemize}
		\item Schlüsselwort um eine Exception zu werfen/auszulösen
		\item Syntax:
		\begin{itemize}
			\item \texttt{throw new \alert{<Exceptionkonstruktor>}(\alert{[Parameter]});}
		\end{itemize}
	\end{itemize}
	\textbf{Eigene Exception}\\
	\begin{itemize}
		\item Man kann eine eigene Excpetion-Klasse schreiben, in dem man von der Klasse \alert{System.Exception} erbt
		\item Dabei ist darauf zu achten, dass man gegeben falls die geerbten Konstruktoren implementiert
	\end{itemize}
\end{frame}

\begin{frame}{Eigene Exceptions}
	\lstinputlisting{resources/05_null_exceptions_schluesselwoerter/exception_throw.cs}
\end{frame}

% ----------------------- Schlüsselwörter ------------------------------
\section{Schlüsselwörter}
\begin{frame}{Schlüsselwörter}
	% static, final, param, var, ref
\end{frame}

\end{document}
