% The Slide Definitions
\input{../templates/course_definitions}

% Author and Course information
% This Document contains the information about this course.

% Authors of the slides
\author{Marc Satkowski, Sascha Peukert}

% Name of the Course
\institute{C\texttt{\#} Kurs}

% Fancy Logo
\titlegraphic{\hfill\includegraphics[height=1.25cm]{../templates/fsr_logo_cropped}}


% Presentation title
\title{Exceptions und Vererbung}
\date{\today}


\begin{document}

\maketitle

\begin{frame}{Gliederung}
	\setbeamertemplate{section in toc}[sections numbered]
	\tableofcontents
\end{frame}

% ----------------------- Dokumentationskommentare ------------------------------
\section{Dokumentationskommentare}
\begin{frame}{Dokumentationskommentare}
	\begin{itemize}
		\item Das Doku-Kommentar beginnt mit 3, anstatt 2 \texttt{\alert{/}}
		\item Beschreiben nur Klassen, Felder/Eigenschaften und Methoden
		\begin{itemize}
			\item Das Kommentar steht dabei immer direkt über den zu kommentierenden Code
		\end{itemize}
		\item Man kann aus diesen eine Dokumentation generieren
		\item Außerdem können einige IDEs (z.B. Visual Studio) daraus eine Kurzbeschreibung der Elemente während des Programmierens schaffen		
	\end{itemize}
\end{frame}

\begin{frame}{Dokumentationskommentare}
	\textbf{Tags}
	\begin{itemize}
		\item Geben an, was genau von dem Code dokumentiert werden soll
		\item 3 wichtige Tags:
		\begin{description}
			\item[summary] Beschreibung des zu kommentierenden Codes\\ Für alle 3 Arten
			\item[param] Beschreibung der Übergabeparameter einer Methode
			\item[returns] Beschreibung des Rückgabewertes einer Methode
		\end{description}
		\item \href{https://msdn.microsoft.com/de-de/library/5ast78ax.aspx}{\alert{Hier}} findet man weitere Tags
	\end{itemize}
\end{frame}

\begin{frame}{Dokumentationskommentare}
	\begin{itemize}
		\item Die genannten Tags am Beispiel einer Methode:
	\end{itemize}
	\lstinputlisting{resources/05_exceptions_und_vererbung/documentation.cs}	
\end{frame}

% ----------------------- Vererbung ------------------------------
\section{Vererbung}
\begin{frame}{Vererbung}
	\begin{itemize}
		\item Ermöglicht das definieren von Klassen die auf anderen basieren
		\item Diese erben die Funktionalitäten der Basisklasse und können diese benutzen oder erweitern
		\item Es wird alles vererbt, doch einige Member sind von dem Erben nicht sichtbar 
		\item Alle Klassen erben von der Klasse \alert{Object}
		\item Klassen können maximal von einer Klasse erben
		\item Die Klasse die Verhalten vererbt heiß \alert{Basisklasse} und die Klasse, die es erbt, heißt \alert{abgeleitete Klasse}
		\item Syntax:
		\begin{itemize}
			\item \texttt{class \alert{<Klassenname>} : \alert{<Basisklasse>}} { }
		\end{itemize}
	\end{itemize}
	\lstinputlisting{resources/05_exceptions_und_vererbung/inheritance.cs}
\end{frame}

\subsection{Polymorphie}
\begin{frame}{Polymorphie}
	\begin{itemize}
		\item Zu dt. Vielgestaltigkeit
		\item Instanzen abgeleitete Klassen können auch immer als Instanzen einer ihrer Basisklassen betrachtet werden
		\item Basisklassen können Methoden und Eigenschaften besitzen, die von den abgeleiteten Klassen überschrieben werden können
		\item Während der Nutzung eines Objekts als Basisklasse wird der Laufzeittyp ermittelt und die überschrieben/veränderten Methoden und Eigenschaften genutzt
		\item Alle nicht überschriebenen Methoden oder Eigenschaften sind identisch mit denen der Basisklasse
	\end{itemize} 
\end{frame}

\subsection{Methoden- \& Eigenschaftsüberschreibung}
\begin{frame}{Methoden- \& Eigenschaftsüberschreibung}	
	\textbf{virtual}
	\begin{itemize}
		\item Erlaubt das Überschreiben einer Methode
		\item Wird an die Methoden einer Basisklasse geschrieben
	\end{itemize}
	\lstinputlisting{resources/05_exceptions_und_vererbung/virtual.cs}
\end{frame}

\begin{frame}{Methoden- \& Eigenschaftsüberschreibung}	
	\textbf{override}\\
	\begin{itemize}
		\item Gibt an, dass es sich um eine Überschreibung einer Methode aus der Basisklasse handelt
		\begin{itemize}
			\item Es darf bloß überschrieben werden, wenn die Methode in der Basisklasse ein \alert{virtual} hat
		\end{itemize}
		\item Wird an die Methoden in der abgeleiteten Klasse geschrieben
	\end{itemize}
	\textbf{new}\\
	\begin{itemize}
		\item Gibt an ob eine neue Methode mit selber Signatur wie in einer Basisklasse erzeugt wird
		\begin{itemize}
			\item Die ursprüngliche Methode wird bei der Nutzung der abgeleiteten Klasse ausgeblendet
		\end{itemize}
	\end{itemize}
\end{frame}

\begin{frame}{Methoden- \& Eigenschaftsüberschreibung}
	\lstinputlisting{resources/05_exceptions_und_vererbung/override_new.cs}
\end{frame}

\subsection{Weitere Schlüsselwörter}
\begin{frame}{Weitere Schlüsselwörter}
	\textbf{sealed}\\
	\begin{itemize}
		\item Modifikator für Klassen und Methoden
		\item Verhindert das von einer Klasse geerbt werden kann
		\item Bei Methoden wird verhindert, dass diese weiter überschrieben werden können
	\end{itemize}
	\lstinputlisting{resources/05_exceptions_und_vererbung/sealed.cs}
\end{frame}

\begin{frame}{Weitere Schlüsselwörter}
	\textbf{as}\\
	\begin{itemize}
		\item Wird als Casting-Operator zwischen vererbten Klassen genutzt
		\item Syntax:
		\begin{itemize}
			\item \texttt{(\alert{<Variable>} as \alert{<Datentyp/Klasse>})}
		\end{itemize}
		\item Nach der Klammer kann auf die Variable als der neue Typ zugegriffen werden
		\item Falls die Umwandlung nicht möglich ist, wird die Operation als \alert{null} ausgewertet
	\end{itemize}	
	\lstinputlisting{resources/05_exceptions_und_vererbung/as.cs}
\end{frame}

\begin{frame}{Weitere Schlüsselwörter}
	\textbf{base}\\
	\begin{itemize}
		\item Ist eine Referenz auf die Member der Basisklasse
		\item Ermöglicht es auf die Methoden- und Eigenschaftsimplementationen der Basisklasse zuzugreifen
	\end{itemize}
	\begin{itemize}
		\item Muss genutzt werden, wenn die Basisklasse einen Konstruktor mit Übergabeparametern hat
		\begin{itemize}
			\item Dabei wird der Konstruktor der Basisklasse vor dem eigenen aufgerufen
		\end{itemize}
		\item Syntax:
		\begin{itemize}
			\item \texttt{\alert{Klasse}(\alert{[Übergabeparameter-B][Übergabeparameter]})\\ : base(\alert{[Übergabeparameter-B]}) \{ \alert{<Code>} \}}
		\end{itemize}
	\end{itemize}
\end{frame}

\begin{frame}{Weitere Schlüsselwörter}
	\lstinputlisting{resources/05_exceptions_und_vererbung/base.cs}
\end{frame}

% ----------------------- Exceptions ------------------------------
\section{Exceptions}
\begin{frame}{Exceptions}
	\begin{itemize}
		\item Zu dt. Ausnahmen
		\item Zeigt an, dass ein außergewöhnlicher oder unerwarteter Programmzustand eingetreten ist
		\item Beinhalten Information über Art und Auftreten des Fehlers
		\item Es gibt verschiedene Arten, wie z.B.:
		\begin{itemize} 
			\item ArgumentException
			\item NullReferenceException	
			\item FileNotFoundException
		\end{itemize}
		\item Exception welche nicht behandelt wurden, führen zum Absturz des Programms
	\end{itemize}
\end{frame}

\subsection{Handling}
\begin{frame}{Handling}
	\textbf{try catch finally}\\
	\begin{itemize}
		\item Wird genutzt um Exceptions aufzufangen und diese zu behandeln
		\item Der im \alert{try}-Block stehende Code wird ausgeführt und bei auftreten einer Exception werden die \alert{catch}-Blöcke ausgeführt
		\item Es kann mehrere \alert{catch}-Blöcke für verschiedene Exception geben
		\begin{itemize}
			\item Die Exception kann im \alert{catch}-Block als Variable genutzt werde
			\item Man geht immer vom spezifischen zum allgemeinem Fehler, damit alle Exceptions gefangen werden können
		\end{itemize}
		\item Der \alert{finally}-Block wird immer ausgeführt, auch wenn keine Exception aufgetreten ist
	\end{itemize}
\end{frame}

\begin{frame}{Handling}
	\lstinputlisting{resources/05_exceptions_und_vererbung/exception_handling.cs}	
\end{frame}

\subsection{Eigene Exceptions}
\begin{frame}{Eigene Exceptions}
	\textbf{throw}\\
	\begin{itemize}
		\item Schlüsselwort um eine Exception zu werfen/auszulösen
		\item Syntax:
		\begin{itemize}
			\item \texttt{throw new \alert{<Exceptionkonstruktor>}(\alert{[Parameter]});}
		\end{itemize}
	\end{itemize}
	\textbf{Eigene Exception}\\
	\begin{itemize}
		\item Man kann eine eigene Excpetion-Klasse schreiben, in dem man von der Klasse \alert{System.Exception} erbt
	\end{itemize}
\end{frame}

\begin{frame}{Eigene Exceptions}
	\lstinputlisting{resources/05_exceptions_und_vererbung/exception_throw.cs}
\end{frame}

\end{document}
