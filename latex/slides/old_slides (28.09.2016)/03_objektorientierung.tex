% The Slide Definitions
\input{../templates/course_definitions}

% Author and Course information
% This Document contains the information about this course.

% Authors of the slides
\author{Marc Satkowski, Sascha Peukert}

% Name of the Course
\institute{C\texttt{\#} Kurs}

% Fancy Logo
\titlegraphic{\hfill\includegraphics[height=1.25cm]{../templates/fsr_logo_cropped}}


% Presentation title
\title{Objektorientierung}
\date{\today}


\begin{document}

\maketitle

\begin{frame}{Gliederung}
	\setbeamertemplate{section in toc}[sections numbered]
	\tableofcontents
\end{frame}

% ----------------------- Objektorientierung ------------------------------
\section{Objektorientierung}
\begin{frame}{Objektorientierung}
	\begin{itemize}
		\item Alles in C\texttt{\#} ist ein Objekt, auch die primitiven Datentypen, wie z.B. int oder bool
		\item Damit kann alles in C\texttt{\#} Felder und Methoden besitzen		
	\end{itemize}
\end{frame}

\subsection{Klassen}
\begin{frame}{Klassen}
	\begin{itemize}
		\item Ist eine Vorlage oder ein Bauplan
		\item Kann Felder, Eigenschaften und Methoden besitzen
		\item Klassen werden durch das Schlüsselwort \alert{class} gekennzeichnet
	\end{itemize}	
	\lstinputlisting{resources/03_objektorientierung/class.cs}
\end{frame}

\subsection{Objekte}
\begin{frame}{Objekte}
	\begin{itemize}
		\item Werden anhand von Klassen erstellt und sind somit Instanzen einer Klasse
		\begin{itemize}
			\item Dies geschieht mit dem Schlüsselwort \alert{new} und einem von der Klasse definierten Konstruktor
		\end{itemize}
		\item Befüllen die Felder einer Klasse mit konkreten Werten
	\end{itemize}
	\lstinputlisting{resources/03_objektorientierung/objects.cs}	
\end{frame}

\subsection{Konstruktor}
\begin{frame}{Konstruktor}
	\begin{itemize}
		\item Sind spezielle Methoden die beim Erstellen einer Instanz einer Klasse aufgerufen werden
		\item Sie besitzen keinen Namen und haben als Rückgabetyp ein Objekt der eigenen Klasse
		\item Sollte sicherstellen, dass eine gültige Instanz einer Klasse erzeugt wird
		\item Falls kein eigener Konstruktor deklariert wurde, existiert immer der Standardkonstruktor (parameterlos und weist allen Feldern Standardwerte zu)
	\end{itemize}
	\lstinputlisting{resources/03_objektorientierung/constructor.cs}	
\end{frame}

\begin{frame}{Konstruktor}
	\textbf{Schlüsselwort new}\\
	\begin{itemize}
		\item Zur Erstellung einer Instanz einer Klasse
		\item Ruft immer einen Konstruktor der jeweiligen Klasse auf
	\end{itemize}
	\lstinputlisting{resources/03_objektorientierung/new.cs}
\end{frame}

\subsection{Felder \& Eigenschaften}
\begin{frame}{Felder \& Eigenschaften}
	\textbf{Felder}\\
	\begin{itemize}
		\item Werden auch Attribute genannt
		\item Sind Variablen eines beliebigen Typs, welche direkt in einer Klasse deklariert wurden und damit dieser Klasse zugehörig sind
	\end{itemize}
	\textbf{Eigenschaften}\\
	\begin{itemize}
		\item Bietet flexible Mechanismen zum Lesen, Schreiben und Berechnen der Werte eines Feldes
		\item Es muss aber nicht zwangsläufig ein Feld abbilden
		\item Besitzen nach dem Namen einen Block in denen ein \alert{get} und/oder \alert{set} definiert werden kann
	\end{itemize}
\end{frame}

\begin{frame}{Felder \& Eigenschaften}
	\textbf{Eigenschaften: get \& set}
	\begin{itemize}
		\item Diese agieren wie Methoden
		\item Sie können einen Körper haben, müssen es aber nicht
		\begin{itemize}
			\item Sobald einer der beiden einen Körper hat, muss der andere auch einen haben (wenn sowohl \alert{get} als auch \alert{set} implementiert werden sollen)
			\item Falls kein eigener Körper definiert wurde, wird die Variable wie ein Feld benutzt
		\end{itemize}
		\item Besitzen eigene Sichtbarkeiten (unterschiedlich vom Feld selbst)
		\item \alert{set} hat dabei einen Übergabeparameter, und zwar \alert{value}
		\begin{itemize}
			\item Dieser ist vom selben Typ, wie die Eigenschaft		
		\end{itemize}		
	\end{itemize}
\end{frame}

\begin{frame}{Felder \& Eigenschaften}	
	\lstinputlisting{resources/03_objektorientierung/field_property.cs}
\end{frame}

\subsection{Zugriff}
\begin{frame}{Zugriff}
	\textbf{Zugriff mit .}\\
	\begin{itemize}
		\item Um auf Inhalte (Methoden, Eigenschaften, ...) eines Objektes zugreifen zu können, nutzt man einen Punkt \alert{\textbf{.}}
		\item Dahinter kommt der Member den man nutzen möchte
	\end{itemize}
	\textbf{this}\\
	\begin{itemize}
		\item Gibt die aktuelle Instanz der Klasse (in dem man gerade ist und arbeitet) als Variable wieder
		\item Nutzen:
		\begin{itemize}
			\item Übergabe der aktuellen Instanz der Klasse in Methoden
			\item Nutzung von Membern, die durch gleiche Kennzeichner ausgeblendet werden (falls Member und Variablen den gleichen Namen haben)
			\item Und andere (wird später erläutert)
		\end{itemize}
	\end{itemize}
\end{frame}

% ----------------------- Zugriffsmodifikatoren/Sichtbarkeiten ------------------------------
\section{Gültigkeit (Scope)}
\begin{frame}{Gültigkeit (Scope)}
	\begin{itemize}
		\item Variablen, Member, Klassen und andere Elementen sind nur innerhalb des Bereichs gültig in denen sie deklariert wurden
		\item Methoden und Felder nur in deren Klassen, Parameter nur in deren Methoden, lokale Variablen innerhalb ihrer Blöcke
	\end{itemize}
	\lstinputlisting{resources/03_objektorientierung/scope.cs}
\end{frame}

% ----------------------- Zugriffsmodifikatoren/Sichtbarkeiten ------------------------------
\section{Zugriffsmodifikatoren/Sichtbarkeiten}
\begin{frame}{Zugriffsmodifikatoren/Sichtbarkeiten}
	\begin{itemize}
		\item Geben an, ob der zugehörige Code von anderen Code-Abschnitten verwendet werden kann
		\item Es gibt mehrere Arten
		\begin{description}
			\item[public] kann von überall zugriffen werden, solange darauf verweist wird
			\item[private] kann nur von Code in der eigenen Klasse benutzt werden
			\item[protected] kann nur von Code in der eigenen und von abgeleiteten Klassen benutzt werden
			\item[internal] kann von überall aus der eigenen Assembly heraus benutzt werden
		\end{description}
	\end{itemize}
\end{frame}

\begin{frame}{Zugriffsmodifikatoren/Sichtbarkeiten}
	\lstinputlisting{resources/03_objektorientierung/access_modifier.cs}
\end{frame}

% ----------------------- namespace und using ------------------------------
\section{namespace \& using}
\begin{frame}{namespace \& using}
	\textbf{namespace}\\
	\begin{itemize}
		\item Werden zur Organisation von Klassen und anderem Code genutzt 
		\item Können beliebig verschachtelt werden
		\item der volle Name einer Klasse beinhaltet die Namen aller Namespaces, in denen die Klasse liegt
		\item Vermeiden von Mehrdeutigen Bezeichnen 
		\begin{itemize}
			\item Zwei Klassen dürfen den selben Namen haben, wenn diese in unterschiedlichen Namespaces liegen		
		\end{itemize}
	\end{itemize}	
	\lstinputlisting{resources/03_objektorientierung/namespace.cs}
\end{frame}

\begin{frame}{namespace \& using}
	\textbf{using}\\
	\begin{itemize}
		\item Ein Schlüsselwort zur Nutzung eines bestimmten \alert{namespace}
		\item Dies erspart die Eingabe des gesamten namespace bei Nutzung einer Klasse
		\item Kann zu merhdeutigen Code führen, wenn 2 Namespaces eingebunden werden, die Elemente mit selben Namen beinhalten
	\end{itemize}
	\lstinputlisting{resources/03_objektorientierung/using.cs}
\end{frame}

% ----------------------- Dokumentationskommentare ------------------------------
\section{Dokumentationskommentare}
\begin{frame}{Dokumentationskommentare}
	\begin{itemize}
		\item Das Doku-Kommentar beginnt mit 3, anstatt 2 \alert{/}
		\item Beschreiben nur Klassen, Felder/Eigenschaften und Methoden
		\begin{itemize}
			\item Das Kommentar steht dabei immer direkt über den zu kommentierenden Code
		\end{itemize}
		\item Man kann aus diesen eine Dokumentation generieren
		\item Außerdem können einige IDEs (z.B. Visual Studio) daraus eine Kurzbeschreibung der Elemente während des Programmierens schaffen		
	\end{itemize}
\end{frame}

\begin{frame}{Dokumentationskommentare}
	\textbf{Tags}
	\begin{itemize}
		\item Geben an, was genau von dem Code dokumentiert werden soll
		\item 3 wichtige Tags:
		\begin{description}
			\item[summary] Beschreibung des zu kommentierenden Codes\\ Für alle 3 Arten
			\item[param] Beschreibung der Übergabeparameter einer Methode
			\item[returns] Beschreibung des Rückgabewertes einer Methode
		\end{description}
		\item \href{https://msdn.microsoft.com/de-de/library/5ast78ax.aspx}{Hier findet man weitere Tags}
	\end{itemize}
\end{frame}

\begin{frame}{Dokumentationskommentare}
	\begin{itemize}
		\item Die genannten Tags am Beispiel einer Methode:
	\end{itemize}
	\lstinputlisting{resources/03_objektorientierung/documentation.cs}	
\end{frame}

\end{document}
