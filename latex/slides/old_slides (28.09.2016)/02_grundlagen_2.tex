% The Slide Definitions
\input{../templates/course_definitions}

% Author and Course information
% This Document contains the information about this course.

% Authors of the slides
\author{Marc Satkowski, Sascha Peukert}

% Name of the Course
\institute{C\texttt{\#} Kurs}

% Fancy Logo
\titlegraphic{\hfill\includegraphics[height=1.25cm]{../templates/fsr_logo_cropped}}


% Presentation title
\title{Grundlagen von C\texttt{\#} - 2}
\date{\today}


\begin{document}

\maketitle

\begin{frame}{Gliederung}
	\setbeamertemplate{section in toc}[sections numbered]
	\tableofcontents
\end{frame}

% ----------------------- Arrays ------------------------------
\section{Arrays}
\begin{frame}{Arrays}
	\begin{itemize}
		\item Sind Auflistung einer beliebigen (aber nach Initalisierung festen) Anzahl von Variablen gleichen Typs
		\item Können über eine Index aufgerufen werden
		\begin{itemize}
			\item Dieser geht von 0 bis (n - 1), wobei n die Anzahl der Elemente des Arrays ist
		\end{itemize}
		\item Können ein- oder mehrdimensional sein
	\end{itemize}
	\textbf{Eindimensionales Array:}\\
	\lstinputlisting{resources/02_grundlagen_2/array.cs}	
\end{frame}

\begin{frame}{Arrays}
	\textbf{Verzweigtes Array:}\\
	\lstinputlisting{resources/02_grundlagen_2/array_of_array.cs}
	\textbf{Mehrdimensionales Array:}\\
	\lstinputlisting{resources/02_grundlagen_2/multi_array.cs}	
\end{frame}

% ----------------------- Kontrollstrukturen ------------------------------
\section{Kontrollstrukturen}
\begin{frame}{Kontrollstrukturen}
	\begin{itemize}
		\item Verändern den Ablauf des Programms
		\item Können Verzweigungen oder Schleifen sein
		\item Benötigen bool'sche Ausdrücke als Abfrage
		\begin{itemize}
			\item Beispiele: bool-Variable/-Methode, Vergleichsoperation, ...
		\end{itemize}
		\item Besitzen immer einen Block, den sie ausführen, wenn die zugehörige Bedingung erfüllt ist
		\begin{itemize}
			\item Falls der Block nur eine Zeile hat, können die geschweiften Klammern weggelassen werden
		\end{itemize}
	\end{itemize}
\end{frame}

% ----------------------- Verzweigung ------------------------------
\section{Verzweigungen}
\subsection{Bedingte Verzweigung (if else)}
\begin{frame}{Bedingte Verzweigung (if else)}
	\begin{itemize}
		\item Prüft die Bedingung des jeweiligen Kopfes
		\begin{itemize}
			\item Die Köpfe werden von oben nach unten durchgegangen
		\end{itemize}
		\item Falls ein Kopf zutrifft, wird der zugehörige Block ausgeführt und die anderen Köpfe werden nicht mehr geprüft
		\item Falls keine der Köpfe zutrifft wird der \alert{else} Block gewählt
		\begin{itemize}
			\item Dieser ist optional
		\end{itemize}
	\end{itemize}
	\lstinputlisting{resources/02_grundlagen_2/if.cs}
\end{frame}

\subsection{Tertiäre-Verzweigung}
\begin{frame}{Tertiäre-Verzweigung}
	\begin{itemize}
		\item Werden genauso genutzt wie eine if-else-Verzweigung
		\item Syntax:
		\begin{itemize}
			\item \texttt{\alert{<Bedingung>} ? \alert{<Dann-Pfad>} : \alert{<Sonst-Pfad>}}
		\end{itemize}
		\item Das \alert{\texttt{?}} steht für \alert{dann}
		\item Das \alert{\texttt{:}} steht für \alert{sonst}
	\end{itemize}		
	\lstinputlisting{resources/02_grundlagen_2/tertiary.cs}
\end{frame}

\subsection{switch case}
\begin{frame}{switch case}
	\begin{itemize}
		\item Hat ähnliches Verhalten wie die bedingte Verzweigung
		\begin{itemize}
			\item \alert{default} verhält sich wie \alert{else}
		\end{itemize}
		\item Jeder \alert{case} prüft die Variable im \alert{switch} auf Gleichheit und führt den darunterliegenden Code bis zum \alert{break} aus
	\end{itemize}
	\lstinputlisting{resources/02_grundlagen_2/switch_case.cs}
\end{frame}

% ----------------------- Schleifen ------------------------------
\section{Schleifen}
\subsection{Kopfgesteuerte Schleife (while)}
\begin{frame}{Kopfgesteuerte Schleife (while)}
	\begin{itemize}
		\item Die Schleife läuft so lange durch deren Körper, wie die Bedingung in den Klammern erfüllt ist
		\item Die Bedingung wird am Anfang geprüft, und falls diese nicht \alert{true} ist, wird der Körper nie ausgeführt
	\end{itemize}
	\lstinputlisting{resources/02_grundlagen_2/while.cs}	
\end{frame}

\subsection{Fußgesteuerte Schleife (do while)}
\begin{frame}{Fußgesteuerte Schleife (do while)}
	\begin{itemize}
		\item Selbe Funktion wie die while-Schleife
		\item Die Bedingung wird hier aber erst am Ende evaluiert
		\item Dementsprechend wird der Körper mindestens einmal ausgeführt, selbst wenn die Bedingung beim ersten Durchlauf falsch sein sollte
	\end{itemize}
	\lstinputlisting{resources/02_grundlagen_2/do_while.cs}	
\end{frame}

\subsection{Zählschleife (for)}
\begin{frame}{Zählschleife (for)}
	\begin{itemize}
		\item Zählt über eine Zählvariable in gewisser Schrittweite, bis die Bedingung nicht mehr erfüllt ist
		\item Syntax:
		\begin{itemize}
			\item \texttt{for( \alert{<Zählvariable>}; \alert{<Bedingung>}; \alert{<Schrittweite>} )\\ \{ \alert{<Code>} \}}
		\end{itemize}
	\end{itemize}
	\lstinputlisting{resources/02_grundlagen_2/for.cs}	
\end{frame}

\subsection{break \& continue}
\begin{frame}{break \& continue}
	\textbf{break}\\
	\begin{itemize}
		\item Beendet die direkt umschließende Schleife, in der die Anweisung auftritt
		\item Beachtet dabei nicht wie viele Schleifendurchläufe es eigentlich noch geben würde
	\end{itemize}
	\textbf{continue}\\
	\begin{itemize}
		\item Beendet den jetzigen Schleifen-Block-Durchlauf und geht zum Kopf der Schleife zurück
		\item Daraufhin kann die Schleife mit dem nächsten Iterationsschritt fortsetzen
	\end{itemize}
\end{frame}

\begin{frame}{break \& continue}
	\textbf{break \& continue:}\\
	\begin{itemize}
		\item Beide können und sollten mit Verzweigungen benutzt werden
	\end{itemize}
	\lstinputlisting{resources/02_grundlagen_2/break_continue.cs}
\end{frame}

\end{document}
