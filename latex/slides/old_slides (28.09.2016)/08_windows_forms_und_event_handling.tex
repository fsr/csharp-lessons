% The Slide Definitions
\input{../templates/course_definitions}

% Author and Course information
% This Document contains the information about this course.

% Authors of the slides
\author{Marc Satkowski, Sascha Peukert}

% Name of the Course
\institute{C\texttt{\#} Kurs}

% Fancy Logo
\titlegraphic{\hfill\includegraphics[height=1.25cm]{../templates/fsr_logo_cropped}}


% Presentation title
\title{Windows Forms \& Event Handling}
\date{\today}


\begin{document}

\maketitle

\begin{frame}{Gliederung}
	\setbeamertemplate{section in toc}[sections numbered]
	\tableofcontents
\end{frame}

\section{Event Handling}
\subsection{Event Handler}
\begin{frame}{EventHandler}
	\begin{itemize}
		\item Ein vordefinierter Delegate
		\item Wird genutzt um Event handling zu betreiben
		\begin{itemize}
			\item Man kann auch einen eigenen nehmen
		\end{itemize}
		\item Gibt \alert{void} zurück
		\item Hat 2 Übergabeparameter:
		\begin{itemize}
			\item \alert{object} (als Sender)
			\item \alert{TEventArgs}
		\end{itemize}
	\end{itemize}
	\lstinputlisting{resources/08_windows_forms_und_event_handling/event_handler.cs}	
\end{frame}

\begin{frame}{EventArgs}
	\begin{itemize}
		\item Gibt dem Event die Möglichkeit zusätzliche Informationen an Empfänger zu übergeben
		\begin{itemize}
			\item Falls keine Informationen zum Event zur Verfügung stehen kann man \alert{EventArgs.Empty} übergeben
		\end{itemize}
		\item Da der EventHanlder generisch ist, kann man seine eigenen \alert{EventArgs} übergeben
	\end{itemize}
	\lstinputlisting{resources/08_windows_forms_und_event_handling/event_args.cs}
\end{frame}

\subsection{event}
\begin{frame}{event}
	\begin{itemize}
		\item Wird genutzt um bei einem Publisher ein Event zu deklarieren
		\item Zusammen mit Attributen von einem Delegatetype genutzt
		\item Kann ebenfalls alle Modifikatoren (wie \alert{virtual} oder \alert{override} nutzen)
		\begin{itemize}
			\item Bei \alert{abstract} muss ein eigener add/remove-Ereignisaccessorblock erstellt werden
		\end{itemize}
	\end{itemize}				
	\lstinputlisting{resources/08_windows_forms_und_event_handling/event.cs}
\end{frame}

\subsection{Mehrere Events}
\begin{frame}{Mehrere Events}
	\begin{itemize}
		\item Man kann verschiedene Methoden einen EventHandler (oder anderen \alert{Delegate}) mit dem add-Ereignisaccessorblock
			\begin{itemize}
				\item \texttt{\alert{+=}}
			\end{itemize}
		\item Entfernt können diese wieder mit dem remove-Ereignisaccessorblock
			\begin{itemize}
				\item \texttt{\alert{-=}}
			\end{itemize}		
		\item Die Methoden werden dann nach einander ausgeführt
	\end{itemize}
\end{frame}

\subsection{Beispiel}
\begin{frame}{Beispiel}
	\lstinputlisting{resources/08_windows_forms_und_event_handling/event_handling_creation.cs}
\end{frame}

\begin{frame}{Beispiel}
	\lstinputlisting{resources/08_windows_forms_und_event_handling/event_handling_usage_1.cs}
\end{frame}

\begin{frame}{Beispiel}
	\lstinputlisting{resources/08_windows_forms_und_event_handling/event_handling_usage_2.cs}
\end{frame}


\end{document}
