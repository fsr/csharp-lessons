% The Slide Definitions
\input{../templates/course_definitions}

% Author and Course information
% This Document contains the information about this course.

% Authors of the slides
\author{Marc Satkowski, Sascha Peukert}

% Name of the Course
\institute{C\texttt{\#} Kurs}

% Fancy Logo
\titlegraphic{\hfill\includegraphics[height=1.25cm]{../templates/fsr_logo_cropped}}


% Presentation title
\title{Grundlagen von C\texttt{\#} - 1}
\date{\today}


\begin{document}

\maketitle

\begin{frame}{Gliederung}
	\setbeamertemplate{section in toc}[sections numbered]
	\tableofcontents
\end{frame}

% ----------------------- Über den Kurs ------------------------------
\section{Über diesen Kurs}
\begin{frame}{Über diesen Kurs}
	\begin{itemize}
		\item 11 Kurseinheiten (Foliensätze) und viele Aufgaben
		\item keine besonderen Programmierkenntnisse vorausgesetzt, aber nützlich
		\item Ressourcen
		\begin{itemize}
			\item Google (C\texttt{\#} meine Frage hier); man landet oft auf der Microsoft Seite
			\item \href{http://stackoverflow.com/}{\alert{Stackoverflow}}
			\item unsere \href{https://github.com/fsr}{\alert{Github-Organisation}}
		\end{itemize}
	\end{itemize}
\end{frame}

\begin{frame}{Über diesen Kurs}
	\centering \normalsize Die Materialien findet ihr hier: \\
	\huge \alert{\href{https://fsr.github.io/csharp-lessons/}{fsr.github.io/csharp-lessons/}}\\	
\end{frame}

% ----------------------- Software ------------------------------
\section{Benötigte Software}
\begin{frame}{Benötigte Software}
	\textbf{.Net-Framework}
	\begin{itemize}
		\item Kann mit Visual Studio installiert werden
		\item oder mit \href{http://www.mono-project.com/}{Mono} für Linux, Windows oder Mac
	\end{itemize}
	\textbf{IDEs}
	\begin{itemize}
		\item \href{https://www.visualstudio.com/}{Visual Studio} für Windows
		\item \href{http://www.monodevelop.com/}{Mono Develop} für Linux, Windows oder Mac
		\item \href{https://www.visualstudio.com/de-de/products/code-vs.aspx}{Visual Code} für Linux, Windows oder Mac
		\begin{itemize}
			\item Benötigt noch eine C\texttt{\#}-Extension (Ctrl+P \texttt{->} ext install C\texttt{\#})
		\end{itemize}
	\end{itemize}
\end{frame}

\section[Exkurs: \newline Architektur von .NET und C\#]{Exkurs:  Architektur von .NET und C\#}
\begin{frame}{Exkurs: Architektur von .NET und C\#}
	\centering
	\href{https://mva.microsoft.com/de-de/training-courses/einstieg-in-c-fr-programmierer-8826?l=MoyWcix2_1304984382}{Link zum Foliensatz "Einleitung und Theorie" des MVA Kurses \\"Einstieg in C\# für Programmierer"}
\end{frame} 

% ----------------------- Hello World ------------------------------
\section{Erstes Programm: Hello World}
\begin{frame}{Hello World}
    \lstinputlisting{resources/01_grundlagen_1/helloworld.cs}   
\end{frame} 

\subsection{Grundlegende Eigenschaften}
\begin{frame}{Grundlegende Eigenschaften}
	\begin{itemize}
		\item Am Ende jedem Befehls steht ein Semikolon (\texttt{\alert{\textbf{;}}})
		\begin{itemize}
			\item Befehle können auch über mehrere Zeilen geschrieben werden
		\end{itemize}
		\item Codeblöcke werden durch geschweifte Klammern gekennzeichnet
		\item Einrückungen sind nicht zwingend aber hilfreich
	\end{itemize}
\end{frame}

% ----------------------- Grundlagen der Sprache ------------------------------
\section{Grundlagen der Sprache}
\begin{frame}{Grundlagen der Sprache}
	\begin{itemize}
		\item C\texttt{\#} ist eine typsichere, objektorientierte Allzweck-Programmiersprache	
		\item Hat auch andere Paradigmen wie:
		\begin{itemize}
			\item imperativ
			\item deklarativ
			\item ereignisorientiert
			\item generisch
			\item ...
		\end{itemize}
	\end{itemize}	
\end{frame}

\subsection{Variablen Deklaration}
\begin{frame}{Variablen Deklaration}
	\textbf{Variablen}\\
		\begin{itemize}
			\item Man kann Variablen gleichzeitig Deklarieren und Initialisieren
			\item Syntax:
			\begin{itemize}			
				\item \texttt{\alert{<Datentyp> <Variablenname>};}
			\end{itemize}
		\end{itemize}
	\lstinputlisting{resources/01_grundlagen_1/variables.cs}
\end{frame}

\subsection{Datentypen}
\begin{frame}{Primitive Datentypen}
	\begin{tabular}{c|l}
		Name & Funktion \\ \hline
		\texttt{int, long} & Ganzzahl "beliebiger" Größe \\
		\texttt{uint, ulong} & Natürliche Zahle "beliebiger" Größe \\
		\texttt{float, double} & Gleitkommazahl "beliebiger" Größe \\
		\texttt{ufloat, udouble} & Positive Gleitkommazahl "beliebiger" Größe \\
        \texttt{decimal} & Präziser Bruch/Ganzzahl mit 29 signifikanten Stellen \\
		\texttt{bool} & Wahrheitswert (\texttt{True}, \texttt{False})\\
        \texttt{char} & Ein einzelnes Zeichen \\
		\texttt{string} & Eine Folge von Zeichen \\
	\end{tabular}
	\begin{itemize}
		 \item Die Variablen mit u (\texttt{unsigned}) als ersten Buchstaben werden nur selten genutzt
	\end{itemize}
\end{frame}

\subsection{Casting}
\begin{frame}{Casting}
	\begin{itemize}
		\item Ist das Umwandeln von einem Datentyp in einen anderen
		\item Klappt bei allen Datentypen die es gibt
		\item Achtung:
		\begin{itemize}
			\item Casting zwischen inkompatiblen Datentypen (z.B. string zu float) kann Fehler hervorrufen
			\item Es kann die Genauigkeiten bei Zahlen verringern
		\end{itemize}
		\item Syntax:
		\begin{itemize}
			\item \texttt{\alert{<Variable>} = (\alert{Variablentyp})\alert{Wert};}
		\end{itemize}
	\end{itemize}
		\lstinputlisting{resources/01_grundlagen_1/casting.cs}
\end{frame}

\subsection{Operatoren}
\begin{frame}{Operatoren}
	\begin{description}
	    \item[mathematisch] \texttt{\alert{+}, \alert{-}, \alert{*}, \alert{/},	\alert{\%}}\\
	    \texttt{\alert{+=}, \alert{-=}, \alert{*=}, \alert{/=}, \alert{++} (Inkrementieren), \alert{--} (Dekrementieren)}
	    \item[vergleichend] \texttt{\alert{<}, \alert{>}, \alert{<=}, \alert{>=}, \alert{==} (Gleichheit), \alert{!=} (Ungleichheit)}
	    \item[logisch] \texttt{\alert{\&\&} (Und), \alert{||} (Oder), \alert{!} (Negation)}
	    \item[bitweise] \texttt{\alert{\&}, \alert{|}, \alert{<<}, \alert{>>}, \alert{\texttt{\^}} (xor), \alert{\texttt{\~}} (invertieren)}
	\end{description}
\end{frame}

\subsection{Kommentare}
\begin{frame}{Kommentare}
	\begin{itemize}
		\item Werden vom Compiler nicht beachtet und dienen nur zum besseren Verständnis des Codes für Programmierer
		\item Außerdem kann aus Dokumentationskommentare eine Dokumentation erstellt werden
	\end{itemize}
	\lstinputlisting{resources/01_grundlagen_1/comments.cs}
\end{frame}

\subsection{Methoden Deklaration}
\begin{frame}{Methoden Deklaration}
	\textbf{Methoden}\\
	\begin{itemize}
		\item Definieren ein gewisses Verhalten, welches durch die Methode aufgerufen werden kann
		\item Diese können noch durch Schlüsselwörter modifiziert werden
		\item Syntax:
		\begin{itemize}			
			\item \texttt{\alert{<Rückgabetyp> <Methodenname>}( \alert{[Übergabeparameter]} ) \\ \{ \alert{<Code>} \}}
		\end{itemize}
	\end{itemize}		
\end{frame}

\begin{frame}{Methoden Deklaration}
	\lstinputlisting{resources/01_grundlagen_1/methods_creation.cs}
	\lstinputlisting{resources/01_grundlagen_1/methods_using.cs}	
\end{frame}

\end{document}
