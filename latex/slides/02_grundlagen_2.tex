% The Slide Definitions
\input{../templates/course_definitions}

% Author and Course information
% This Document contains the information about this course.

% Authors of the slides
\author{Marc Satkowski, Sascha Peukert}

% Name of the Course
\institute{C\texttt{\#} Kurs}

% Fancy Logo
\titlegraphic{\hfill\includegraphics[height=1.25cm]{../templates/fsr_logo_cropped}}


% Presentation title
\title{Grundlagen von C\texttt{\#} - 2}
\date{\today}


\begin{document}

\maketitle

\begin{frame}{Gliederung}
	\setbeamertemplate{section in toc}[sections numbered]
	\tableofcontents
\end{frame}

% ----------------------- Arrays ------------------------------
\section{Arrays}
\begin{frame}{Arrays}
	\begin{itemize}
		\item Sind Auflistung einer beliebigen (aber nach Initalisierung festen) Anzahl von Variablen gleichen Typs
		\item Können über eine Index aufgerufen werden
		\begin{itemize}
			\item Dieser geht von 0 bis Größe - 1
		\end{itemize}
		\item Können ein- oder mehrdimensional sein
		\item Ermöglicht schnellen Zugriff
	\end{itemize}
	\textbf{Eindimensionales Array:}\\
	\lstinputlisting{resources/02_grundlagen_2/array.cs}	
\end{frame}

\begin{frame}{Arrays}
	\textbf{Array von einem Array:}\\
	\lstinputlisting{resources/02_grundlagen_2/array_of_array.cs}
	\textbf{Mehrdimensionales Array:}\\
	\lstinputlisting{resources/02_grundlagen_2/multi_array.cs}	
\end{frame}

% ----------------------- Kontrollstrukturen ------------------------------
\section{Kontrollstrukturen}
\begin{frame}{Kontrollstrukturen}
	\begin{itemize}
		\item Verändern den Ablauf des Programms
		\item Können Verzweigungen oder Schleifen sein
		\item Benötigen bool'sche Ausdrücke als Abfrage
		\begin{itemize}
			\item Beispiele: bool-Variable/-Methode, Vergleichsoperation, ...
		\end{itemize}
		\item Besitzen immer einen Block, den sie ausführen, wenn die zugehörige Bedingung erfüllt ist
		\begin{itemize}
			\item Falls der Block nur eine Zeile hat, können die geschweiften Klammern weggelassen werden
		\end{itemize}
	\end{itemize}
\end{frame}

% ----------------------- Verzweigung ------------------------------
\section{Verzweigungen}
\subsection{if}
\begin{frame}{if}
	\begin{itemize}
		\item Prüft die Bedingungen der Köpfe von oben nach unten
		\item Falls eine zutrifft, wird der zugehörige Block ausgeführt und die anderen Bedingen werden nicht mehr geprüft
		\item Falls keine der Bedingungen zutrifft wird der \alert{else} Block gewählt
	\end{itemize}
	\lstinputlisting{resources/02_grundlagen_2/if.cs}
\end{frame}

\subsection{Tertiäre-Verzweigung}
\begin{frame}{Tertiäre-Verzweigung}
	\begin{itemize}
		\item Werden genauso genutzt wie eine if-else-Verzweigung
		\item Syntax:
		\begin{itemize}
			\item \texttt{\alert{<Bedingung>} ? \alert{<Dann-Pfad>} : \alert{<Sonst-Pfad>}}
		\end{itemize}
		\item Das \alert{\texttt{?}} steht für \alert{dann}
		\item Das \alert{\texttt{:}} steht für \alert{sonst}
	\end{itemize}		
	\lstinputlisting{resources/02_grundlagen_2/tertiary.cs}
\end{frame}

\subsection{switch case}
\begin{frame}{switch case}
	\begin{itemize}
		\item Verhält sich wie die if-Verzweigung
		\begin{itemize}
			\item \alert{default} verhält sich wie \alert{else}
		\end{itemize}
		\item Jeder \alert{case} prüft die Variable im \alert{switch} auf Gleichheit und führt den darunterliegenden Code bis zum \alert{break} aus
	\end{itemize}
	\lstinputlisting{resources/02_grundlagen_2/switch_case.cs}
\end{frame}

% ----------------------- Schleifen ------------------------------
\section{Schleifen}
\subsection{while}
\begin{frame}{while}
	\begin{itemize}
		\item Die Schleife läuft so lange durch deren Körper, wie die Bedingung in den Klammern erfüllt ist
	\end{itemize}
	\lstinputlisting{resources/02_grundlagen_2/while.cs}	
\end{frame}

\subsection{do while}
\begin{frame}{do while}
	\begin{itemize}
		\item Selbe Funktion wie die while-Schleife
		\item Einziger Unterschied ist, dass diese Schleife auf jeden Fall einmal ihren Körper durchläuft, bevor sie die Bedingung überprüft
	\end{itemize}
	\lstinputlisting{resources/02_grundlagen_2/do_while.cs}	
\end{frame}

\subsection{for}
\begin{frame}{for}
	\begin{itemize}
		\item Zählt über eine Zählvariable in gewisser Schrittweite, bis die Bedingung nicht mehr erfüllt ist
		\item Syntax:
		\begin{itemize}
			\item \texttt{for( \alert{<Zählvariable>}; \alert{<Bedingung>}; \alert{<Schrittweite>} )\\ \{ \alert{<Code>} \}}
		\end{itemize}
	\end{itemize}
	\lstinputlisting{resources/02_grundlagen_2/for.cs}	
\end{frame}

\subsection{foreach}
\begin{frame}{foreach}
	\begin{itemize}
		\item Läuft durch ein/e Array/Collection an Elementen eines Typen durch 
		\item Weist jeden Durchlauf ein neues Element aus dem Array/der Collection der Variable zu 
		\item Syntax:
		\begin{itemize}
			\item \texttt{foreach( \alert{<Datentyp> <Variablenname>} in \alert{<Aufzählung>} )\\ \{ \alert{<Code>} \}}
		\end{itemize}
	\end{itemize}
	\lstinputlisting{resources/02_grundlagen_2/foreach.cs}	
\end{frame}

\subsection{break \& continue}
\begin{frame}{Schleifen}
	\textbf{break}\\
	\begin{itemize}
		\item Beendet die direkt umschließende Schleife, in der die Anweisung auftritt
		\item Beachtet dabei nicht wie viele Schleifendurchläufe es eigentlich noch geben würde
	\end{itemize}
	\textbf{continue}\\
	\begin{itemize}
		\item Beendet den jetzigen Schleifen-Block-Durchlauf und geht zum Kopf der Schleife zurück
		\item Daraufhin kann die Schleife mit dem nächsten Iterationsschritt fortsetzen
	\end{itemize}
\end{frame}

\begin{frame}{Schleifen}
	\textbf{break \& continue:}\\
	\begin{itemize}
		\item Beide können und sollten mit Verzweigungen benutzt werden
	\end{itemize}
	\lstinputlisting{resources/02_grundlagen_2/break_continue.cs}
\end{frame}

\end{document}
